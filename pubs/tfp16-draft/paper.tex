% This is LLNCS.DEM the demonstration file of
% the LaTeX macro package from Springer-Verlag
% for Lecture Notes in Computer Science,
% version 2.4 for LaTeX2e as of 16. April 2010
%
\documentclass{llncs}
%
\usepackage{makeidx}  % allows for indexgeneration
%
% mkd for preliminary paper notes
\usepackage{enumitem} 
\usepackage[dvipsnames]{xcolor} % for parenthetical notes while paper is in progress
\usepackage{comment} % comment environment
\usepackage{hyperref} 

\renewcommand{\floatpagefraction}{0.8}
%
\begin{document}
%
%mkd \frontmatter          % for the preliminaries
%
\pagestyle{headings}  % switches on printing of running heads
%
%mkd \mainmatter              % start of the contributions
%
\title{Automatic Parallelization and Transparent Fault Tolerance\\
       (\emph{draft})}
%
\titlerunning{Automatic Parallelization}  % abbreviated title (for running head)
%                                     also used for the TOC unless
%                                     \toctitle is used
%
\author{Kei Davis\inst{1} \and
        Dean Prichard\inst{1} \and
        David Ringo\inst{1,2} \and
        Loren Anderson\inst{1,3} \and
        Jacob Marks\inst{1,4}%\thanks{Ringo, Anderson, and Marks are undergraduate students.}
}
%
\authorrunning{Kei Davis et al.} % abbreviated author list (for running head)
%
%%%% list of authors for the TOC (use if author list has to be modified)
%\tocauthor{Ivar Ekeland, Roger Temam, Jeffrey Dean, David Grove,
%Craig Chambers, Kim B. Bruce, and Elisa Bertino}
%
\institute{Los Alamos National Laboratory, Los Alamos, NM, USA\\
\email{kei.davis@lanl.gov},\\ WWW home page:
\texttt{http://ccsweb.lanl.gov/\homedir kei/}
\and
University of New Mexico,
Albuquerque, NM, USA
\and
North Dakota State University,
Fargo, North Dakota, USA
\and
New Mexico Institute of Mining and Technology,
Socorro, New Mexico, USA
}

\maketitle              % typeset the title of the contribution

% moronic author listing code increments footnote counter
\setcounter{footnote}{0}

\begin{abstract}

A claimed trend is an increasing appreciation and practice of functional
programming in scientific computing, and in particular strict (by default)
functional programming, in various guises.  Besides making large programs
easier to reason about, this enables automatic parallelization to various
degrees.  A second claimed trend in both scientific computing and functional
programming is the development of mechanisms for transparent fault tolerance.
Our project goal is the demonstration of a light-weight, higher-order,
polymorphic, pure functional language implementation in which we can
experiment with automatic parallelization strategies, varying degrees of
strictness, and mechanisms for transparent fault tolerance.
%
We do not consider speculative evaluation, or semantic strictness inferred by
program analysis, so potential parallelism is dictated by the specified degree of
strictness.

\hspace{0.2in} Our development has been greatly informed by the extant
literature and we indicate both where we follow, and deviate from,
these sources.
%
\\
{\color{OliveGreen} % \bf
\emph{Draft paper notes:}
\begin{enumerate}[before=\itshape]
\item We are reporting work in progress.  At the time of writing we have a
  serial implementation, a primitive proof-of-concept parallel
  implementation, and are currently developing a more realistic parallel
  runtime.  
%We expect to have preliminary results from our new parallel
%implementation by the time of the symposium.  
The fault-tolerance mechanisms remain future work.

\item Hopefully in the spirit of a student-friendly symposium, this draft is
  liberally annotated with indicators of student contributions, student
  preparation, the influence of their involvement on our approach, and other
  incidental and ancillary comments.  A final paper would of course be cleaned
  up and potentially considerably shortened.

\item While \textcolor{OrangeRed}{this does not qualify as a student paper}, students
  (all undergraduate) have contributed considerably to this effort, and to the
  writing of this paper, and will continue to do so.  If this paper is
  accepted we plan to have one of these students present it at TiFP'16.
\end{enumerate}
}

\keywords{Strict pure functional programming, automatic parallelization,
  transparent fault tolerance, Haskell, STG, GHC, GHC Core.}

\end{abstract}

\section{Background and Motivation}
The practice of so-called high-performance scientific computing (HPC) is,
overall, highly conservative with respect to change.  Even as the USA and
other governments commit to a push for exascale computing ($10^{18}$ FLOPS in
a single system) within the next
decade~\cite{Exascale-roadmap,EUexascale,Exascale-org} a respected
member of the HPC community recently stated publicly that ``Fortran is
essential for exascale programming''~\cite{Heroux16}.  Existing US government
laboratory HPC codes run to hundreds of thousands of lines of Fortran each and
complete rewrites are infeasible.\footnote{The codes of our lab's British
  counterpart are similar.}  C++ has become more common for new code starts,
but their nature remains the same: huge codes with fair to non-existent
encapsulation of side effects, with multiple levels of parallelism sometimes
poorly abstracted, especially at the thread (shared-memory) and computational
accelerator (e.g.\ GPU) levels.  Thread-level parallel efficiency can be good
when the parallel model is bulk synchronous; tasking models are generally
avoided unless abstracted by, e.g., a C++ template or runtime library, because
of the programming complexity.

\subsection{A trend: strict, pure functional programming in HPC and FP}
Interest in functional programming for HPC in general is not new as evidenced
by the Workshop on Functional High Performance Computing, held in conjunction
with the International Conference on Functional Programming, now in its fifth
consecutive year.  A more precise characterization of the trend is that there
is a small but growing understanding by practicing computational scientists
that \emph{pure functional semantics}, in some form, is essential to reining
in the complexity of ever-evolving scientific codes and informing the designs
of new ones, and that strict semantics tend to be preferred.  Such semantics
are expressed in a number of ways as highlighted following.

\subsubsection{Rely on programmer discipline.}
Legacy Fortran scientific codes likely represent worst practices in state
encapsulation, with most data global and accessible to all; indeed, Fortran
was designed to make this easy via \texttt{COMMON} blocks---aggregates of
global variables---which further obfuscate program meaning by allowing
arbitrary naming of a block's variables at a subroutine/function granularity,
and requiring that entire blocks be brought into scope.  It is almost ironic,
then, that of the mainstream high-performance scientific programming languages
(Fortran, C, and C++), Fortran~95 was the first to introduce the \texttt{pure}
function qualifier, and with enforcement by the compiler.  Otherwise it is
entirely up to the programmer to make functions composable and thread safe.
%Here at LANL the use of pure functions (among other techniques for reducing
%global data dependencies) is being used to good effect to clean up some legacy
%code bases.

\subsubsection{Obey constraints prescribed by a parallel runtime system.}
There are a number of parallel runtime systems that encourage (but cannot
enforce) a pure functional style.  We highlight Stanford's Legion
runtime as an example~\cite{Bauer12}.

Using Legion, all non-function-local variables, or \emph{regions}, must be
requested from the runtime system.  Regions may be shared among \emph{tasks}
(essentially C++ functions), and each task is prescribed access privileges
(read, write, read/write, etc.) and coherency requirements for each region to
which it will have access.  Other than registered access to regions, tasks
must not access non-constant global data.  Thus the inputs and outputs of
tasks are exactly known to the runtime system.  In serial execution these
access requirements are notionally superfluous, but in parallel execution the
runtime can dynamically calculate the data dependency graph and relax the
programmatically defined serial order of task execution to a partial, and
therefore parallelizable, order.  Very high parallel efficiency has been
demonstrated on extremely large computing systems~\cite{Bauer14}.\footnote{We
  note that the PI of the Legion project was previously a co-developer of a
  strict, arguably pure functional language
  implementation~\cite{AikenFL,FLreport89}, in collaboration with John Backus
  as a refinement of his FP language~\cite{Backus:1978}.}

\subsubsection{Use a pure functional language.}
Twenty years ago the US Department of Energy laboratory complex was
presented with the most highly parallel (and thereby performant)
implementation of a pure functional language to date, to no
effect whatsoever~\cite{Davis96}.\footnote{In 1996, on the LANL Thinking Machines
CM-5, the Top 1 supercomputer in 1993~\cite{top1:1993}.}
Today, however, there is a small number of laboratory HPC practitioners
who are developing an appreciation for pure functional programming in Haskell.
However, they see their needs as different than other Haskell users: performance
is paramount, array/vector
%~\cite{vector} 
is a (perhaps \emph{the}) primary data
structure, and non-strictness or laziness is more of a nuisance---even a
serious hindrance---than a convenience, much less ever algorithmically
essential in the sense of Bird's \emph{repmin}~\cite{Bird84},
Johnsson's general attribute grammars~\cite{Johnsson87}, 
or other instances of \emph{tying the knot}~\cite{tying-the-knot}.

A common complaint among these aspiring high-performance functional
programmers is the need to get strictness just right: to remove space leaks,
achieve competitive performance, get I/O and inter-process communication to work
as they expect, and so on.  More specifically, they expect to be able to
reason about space and time usage, and order of evaluation, using their
existing mental models. While they are happy to write code that is strongly
reminiscent of the numerical examples in Hughes' classic piece (whether
they've read it or not)~\cite{Hughes89}, they would also be happy to write in
a slightly modified style that does not require non-strictness 
%
and so that their programs behave as they expect.
%---after all, that's how they've been thinking their entire careers.

Much anecdotal evidence suggests that many others have similar difficulties
with semantics that are lazy (or non-strict) by default, with the most telling
recent concrete evidence being the push for the \texttt{Strict} and
\texttt{StrictData} language extensions newly available in GHC 8.0.  Quoting
the documentation:
\begin{quote}
  High-performance Haskell code (e.g.\ numeric code) can sometimes be littered
  with bang patterns, making it harder to read. The reason is that laziness
  isn't the right default in this particular code, but the programmer has no
  way to say that except by repeatedly adding bang
  patterns~\cite{strict-strictdata}.
\end{quote}
It is a fact that \emph{HPC practitioners are generally not computer
  scientists}, they are domain scientists (physicists, chemists, etc.) or
computational scientists specialized in numerical or numerically-oriented
algorithmic methods (e.g., Lagrangian, Eulerian, adaptive mesh refinement).
The concept of a space leak, for example, caused by other than a missing
explicit deallocation statement (C \texttt{free}, Fortran \texttt{DEALLOCATE}, 
etc.) is difficult to understand and reason about.
%and not something they care to need to understand.
%Hence, we eschew explicit strictness annotations for varying uniform degrees
%of strictness.

More generally, \emph{task-based} parallelism is becoming increasingly popular
for HPC, often in conjunction with a higher-level (typically inter-node) bulk
synchronous model usually implemented with the Message Passing Interface
(MPI)~\cite{MPI}\@.  In the imperative world, example systems include the
aforementioned Legion, Intel's Threading Building Blocks~\cite{Reinders:2007},
and the HPX C++11/14 extension~\cite{HPX}, all of which also seek to
make parallelism automatic and transparent.  While the concept of task is
somewhat vague, the general idea is a unit of computation with no or
minimal/constrained side effects.  A function in a pure functional language
could be regarded as the essence of the concept of a task.  All of these
systems are of course in languages with strict function-call semantics.

\subsection{Another trend: transparent fault tolerance in HPC and FP}
In the quest for computing platforms approaching exascale, it is widely
recognized that new mechanisms for fault tolerance will need to be built into
the software stack.  Current practice in HPC is checkpoint/restart, wherein at
various points in time sufficient global program state is dumped to backing store
such that the program, in case of a fault, can be (manually) restarted from
the last stored state.  Most implementations are explicit, though there has
been some success with transparent (invisible to the programmer) systems.
Manually wiring checkpoint/restart into a large-scale application is not only
burdensome and error-prone, it can severely warp the engineering of the
application because it must be designed such that consistent, essential state
can be readily captured.  Finally, the scalability of global
checkpoint/restart is reaching its end as data transfer to backing store
becomes an increasing bottleneck~\cite{Daly:2006,DOE:2009}.

In the imperative world side-effect-free tasks could be safely restarted after
failure, needing only their well-defined inputs and not arbitrary access to
mutable global state.  The Legion system is being augmented with this
capability, and in such a way that it will be largely transparent to the
programmer.  Ericsson's quasi-functional language Erlang and its runtime
system were designed for fault tolerance~\cite{Cesarini:2009}.  It has long
been recognized that the side-effect-free (or revocably side-effecting)
parts of a functional program could be safely replicated or restarted.  The
recent work of Stewart impressively demonstrates this, in a pure functional
setting, on a large-scale distributed-memory system~\cite{Stewart:2013}.

We claim, then, that there is a trend, in HPC at least, towards
\emph{strict-by-default pure functional programming} and \emph{automatic
parallelism}.
%whether as imposed by programmer discipline, a parallel runtime system, or the
%language itself (e.g.\ strict Haskell).
Second, we claim that there is a trend in high-performance computing, and
functional programming, towards \emph{transparent fault tolerance}.

\subsection{Strictness and Parallelism}
%It is a trivial observation that given strict functional semantics, wherein
%function arguments may be safely evaluated before function evaluation, it is
%safe to evaluate the arguments, and the function, in parallel, and that this
%principle applies recursively in the (dynamic) expression tree (or graph).

Given strict functional semantics wherein function arguments may be safely
evaluated before function evaluation or call, it is safe to evaluate the
arguments, and the function, in parallel, and that this applies recursively in
the (dynamic) expression tree (or graph), and similarly for data constructors
as functions.

\section{Project and Goals}
Our project is the light-weight implementation of a pure, higher-order,
polymorphic, functional language and runtime system with which we can
experiment with automatic parallelization strategies with varying degrees of
language strictness, and secondarily, with mechanisms for transparent fault
tolerance.  Light-weight has several implications.  First and foremost it
means that \emph{we are not attempting to compete with GHC in any
  respect}---succumbing to the temptation of recreating a significant subset
of GHC language, type system, or runtime features, or significant code
optimization, would simply divert us from our central goals.\footnote{We
  estimate that we will have, to within an order of magnitude, one person-hour
  of total available development time for each person-year that GHC currently
  represents, or in other units, one second for each hour.}
For another, it means that we are not undertaking significant morphing of GHC
itself: because this is in large part an undergraduate student project it must
be kept as simple as reasonably possible.

Our initial requirements are that
\begin{itemize}
\item The implementation must be feasible in terms of available effort;
\item The language must be strongly typed, implicitly or explicitly.  More
  specifically, the language should explicitly support both Hindley-Milner
  polymorphism and unboxed types~\cite{Jones:1991};
\item Straightforward direct linkage with C, preferably without a sophisticated
  foreign function interface, is essential;
  \item The language should be \emph{currying-friendly} as described by Marlow
and Peyton Jones~\cite{Marlow:2004};
\item Choice of degree of strictness should be easily selected;
\item Sharing, in the sense of what \emph{lazy} means vs.\ merely non-strict,
  should be preserved for all degrees of strictness;
\item The generated code is C;
\item A certain consistency in tail calling in C must be achieved.
\end{itemize}
%
For feasibility we restrict ourselves to implicit parallelism on shared-memory
multiprocessors.  The requirement for tail calling is not primarily for
efficiency, rather for an efficient parallel implementation.  Achieving
this, and thereby sufficiently limiting the use of the native C stack, will be
discussed at some length.

A minor, informal goal is to gain a sense of how often relatively
inexperienced functional programmers make essential use of
non-strictness and laziness.
%\footnote{The starting experience of our student interns to
%  date has been normalized by having them work through the entirety of Hutton's \emph{Programming
%    in Haskell} textbook~\cite{Hutton-book} as baseline preparation,
%  substituting his article \emph{Higher-order Functions for
%    Parsing}~\cite{Hutton-parsing:1992} for the corresponding material in the
%  book.}

\section{Design and Implementation}
Our wish list has much in common with the GHC Haskell implementation, and
indeed GHC is part of our master plan.  In brief, the phases of the GHC
compiler are (1) parsing and type checking; (2) transformation to the GHC
\emph{Core} intermediate representation~\cite{Sulzmann:2007,ghc-core}; (3)
analysis and transformation on Core; (4) transformation of Core to the
\emph{STG} intermediate representation~\cite{PJ-stockhardware}; and (5)
transformation of STG to machine code via various imperative
representations (Cmm, LLVM, C, native code generation).

In short, our plan is to use GHC as our front-end to generate one of its
intermediate forms, STG or Core, then escape to our system from there.  There
are various possible ways to accomplish the this: using GHC's option to dump
STG or Core as text, hacking GHC itself, and, possibly, using GHC as a
library~\cite{ghc-as-library}.  Initially we start with our own STG.
%, with bridging the gap with GHC scheduled to commence quite soon.

\subsection{STG Language}

\setlength{\tabcolsep}{5pt}
\begin{figure}
\centering
\footnotesize % tiny scriptsize footnotesize small
\begin{tabular}{r r c l l}
Variable     & $f,\ x$        &     &                                              & Initial lower-case letter \\
Constructor  & $C$            &     &                                              & Initial upper-case letter \\
Literal      & $\mathit{lit}$ & ::= & $i\ |\ d$                                    & Integral or floating point literal \\
Atom         & $a$            & ::= & $\mathit{lit}\ |\ x$                         & \\ % Function, constructor args are atoms \\
\\
Expression   & $e$            & ::= & $a$                                          & Atom \\
             &                & $|$ & $f\ a_1\dots a_n$                            & Application, $n\ge 1$ \\
             &                & $|$ & $\oplus\ a_1\dots a_n$                       & Saturated primitive operation, $n\ge 1$ \\

             &                & $|$ & \texttt{let \{}                              & Recursive let, $n\ge 1$ \\
             &                &     & \texttt{ } $\mathit{odecl}_1$ \texttt{;} \\
             &                &     & \hspace{0.2in} $\dots$ \texttt{;} \\
             &                &     & \texttt{ } $\mathit{odecl}_n$ \\
%             &                &     & \texttt{ } $x_1$ \texttt{=} $\mathit{obj}_1$ \texttt{;} \\
%             &                &     & \hspace{0.2in} $\dots$ \texttt{;} \\
%             &                &     & \texttt{ } $x_n$ \texttt{=} $\mathit{obj}_n$ \\
             &                &     & \texttt{\} in} $e$  \\

             &                & $|$ & \texttt{case} $e$ \texttt{of} $x$ \texttt{\{}  & Case expression, $n \ge 1$\\
             &                &     & \texttt{ } $\mathit{alt}_1$ \texttt{;} \\
             &                &     & \hspace{0.2in} $\dots$ \texttt{;} \\
             &                &     & \texttt{ } $\mathit{alt}_n$ \texttt{\}} \\
%            &                &     & \texttt{\}} \\
\\
Alternatives & $\mathit{alt}$ & ::= & $C\ x_1\dots x_n$ \texttt{->} $e$            & Pattern match, $n \ge 0$ \\
             &                & $|$ & \texttt{\_ ->} $e$                           & Default \\
\\
Heap objects & $\mathit{obj}$ & ::= &\texttt{FUN} $x_1\dots x_n$ \texttt{->} $e$   & Function definition, arity $=n\ge 1$ \\
             &                & $|$ &\texttt{CON} $C\ a_1\dots a_n$                & Saturated constructor, $n \ge 0$ \\
             &                & $|$ &\texttt{THUNK} $e$                            & Thunk---explicit deferred evaluation \\
             &                & $|$ &\texttt{PAP} $f\ a_1\dots a_n$               & Partial application \\
             &                & $|$ & $\mathit{BLACKHOLE}$                         & Evaluation-time black hole \\
\\
Object decl. & $\mathit{odecl}$ & ::=  & $x = \mathit{obj}$                           & Simple binding \\
\\
%%%Constructor defn. & $\mathit{con}$  & ::= & $C\ \mathit{type}_i$ & $i \ge 0$ \\
%%%\\
%%%Datatype defn. &  $\mathit{ddecl}$ & ::= & \texttt{data} [\texttt{unboxed}] & User-defined data type  \\
%%%               &                   &     & $C\ x_i =$                       & $i \ge 0, n > 0$         \\
%%%               &                   &     & $\mathit{con}_1 | \dots |  \mathit{con}_n$ \\
%%%\\
%%%Program      & $\mathit{prog}$ & ::= & $\mathit{(o|d)decl}_1$ \texttt{;}            & Object and data defns, \\
%%%             &                 &     & \texttt{ } $\dots$ \texttt{;}                & distinguished \texttt{main}\\
%%%             &                 &     & $\mathit{(o|d)decl}_n$ & 
Program      & $\mathit{prog}$ & ::= & $\mathit{odecl}_1$ \texttt{;}            & Object and data defns, \\
             &                 &     & \texttt{ } $\dots$ \texttt{;}                & distinguished \texttt{main}\\
             &                 &     & $\mathit{odecl}_n$ & 
%Program      & $\mathit{prog}$& ::= & $f_1\ =\ \mathit{obj}_1$ \texttt{;}          & $n \ge 1$, distinguished \texttt{main}\\
%             &                &     & \texttt{ } $\dots$ \texttt{;} \\
%             &                &     & $f_n\ =\ \mathit{obj}_n$

\end{tabular}
\caption{STG syntax}
\label{fig:STGsyntax}
\end{figure}

Our incarnation of STG, as shown in Figure~\ref{fig:STGsyntax}, is meant to
evoke both an abstract and concrete syntax.  This formulation and presentation
is adapted from Marlow and Peyton Jones' \emph{Eval/Apply}
paper~\cite{Marlow:2004}, to which we will subsequently refer as \emph{EA}\@.
A \emph{BLACKHOLE} object, signifying a \emph{THUNK} being evaluated, is only
a run-time entity and not part of the language.  Omitted is the syntax for
algebraic data type definitions which is given in
Appendix~\ref{app:minihaskell}.

With a few exceptions the semantics of STG is what a Haskell programmer
familiar with unboxed types would expect for a higher-order, polymorphic, pure
functional language.  Points to note are, first, that \texttt{case} is strict
in the scrutinee (the \emph{e} in \texttt{case} \emph{e} \texttt{of}), so \texttt{case} is
similar to \texttt{pseq} (as opposed to \texttt{seq}) in Haskell, that is, the
scrutinee is evaluated first.  Second, constructors are not functions, rather
they are named in the construction (heap allocation) of a data object.  More
generally, the operational notion of allocating a heap object is explicit:
this is the operational semantics of \texttt{let} bindings.  Numeric literals
are considered constructors in pattern matching.
%As such a variable of unboxed type cannot be let-bound; instead it can be
%bound with \texttt{case}, e.g., \texttt{case e of x\# \{ \_ -> e \}}
%where the \texttt{\#} suffix is merely a convention to signify that
%\texttt{x\#} has unboxed type.  
Finally, deferred evaluation is explicit:
a deferred expression evaluation is encoded as a \emph{THUNK}.


\subsection{Degrees of strictness}
First some words about terminology.  For all degrees of strictness, sharing
(as implied by lazy but not by non-strict) of heap-allocated objects is
preserved so we will not use the term lazy and simply refer to
strictness properties.  As is common we will use strict in an operational
sense, meaning that argument(s) are evaluated before being passed in a
function call, rather than the denotational sense, unless stated otherwise.
Similarly, when referring to language semantics we refer by default to the
operational rather than denotational semantics.

Greater strictness implies more opportunities for automatic parallelism,
possible losses or gains in space and time efficiency, and less expressiveness.  We
seek to explore the interplay of all of these in various ways.

GHC 8.0 gives three defaults: conventional non-strict, constructor strict, and
function- and constructor strict, but there are possible variations.  The
obvious variation is function, but not constructor, strictness.  In Haskell
the distinction between constructors and functions is somewhat blurred
(outside of data definitions and pattern matching) but in STG the distinction
is clear.

Other possible variations concern the notional arity of the underlying
function being applied.  This is readily explained in terms of the operation
of the STG abstract machine.  Our partial description is based on that given
in \emph{EA}.

In STG the arity of a function is defined as the number of formal parameters
in the function definition, thus \texttt{id x = x} has by definition arity one
regardless of the type of the argument it might be applied to (e.g., some
function type).  We distinguish \emph{application}, e.g., \texttt{f x y},
which is a syntactic notion, from actual function \emph{call}. Thus \texttt{id
  id x} is an application of \texttt{id} to two arguments, but the underlying
\emph{FUN} (user- or system- defined function) when called is given exactly
one argument (the first one).

In the eval-apply model, given application \texttt{f x y}, in the general case, with standard
(non-strict) semantics, \texttt{f} is first evaluated and the
arity of the underlying \emph{FUN} or \emph{PAP} (partial application of a
\emph{FUN}) is determined---in the case of \emph{PAP} the arity is that of the
underlying \emph{FUN} less the number of arguments previously provided.  If
the arity is greater than two a new \emph{PAP} is created with arguments
\texttt{x} and \texttt{y} inserted and the new \emph{PAP} returned.  If the
arity is equal to two the underlying \emph{FUN} is tail-called (jumped to)
with any previous arguments (in case of \emph{PAP}) and new arguments
\texttt{x} and \texttt{y}.  If the arity is less than two (it must be at least
one, so one in this case) the underlying \emph{FUN} is called with any
previous arguments (in case of \emph{PAP}) and new argument \texttt{x}, then
the object returned applied to remaining argument \texttt{y}.

Given application \texttt{f x y}, when might \texttt{x} and \texttt{y} be
evaluated?  In the non-strict semantics evaluation is on demand, after
\texttt{f} has been evaluated, the underlying \emph{FUN} actually called
(i.e., not just partially applied and wrapped up in a \emph{PAP}), and one or
both of those arguments possibly forced in subsequent evaluation, assuming
that they weren't forced in the evaluation of \texttt{f} itself.

In a more strict semantics one could devise various evaluation orders, but as far
as distinguishing termination properties it seems that these come
down to two reasonable possibilities, namely whether on construction of a
\emph{PAP} the (new) arguments are forced or not.  In other words, the
question is whether application is strict, or function call is strict.  These
are in fact distinguishable because (in Haskell, for example) the first
argument to \texttt{seq} could evaluate to a \emph{PAP}.

This then gives three reasonable operational modes which are strictly ordered
in terms of termination properties: 
non-strict, function strict, and application strict.
% non-strict; function strict, in which only
%those arguments needed by a function call are evaluated; and application
%strict, in which all arguments of an application are evaluated.  
The other dimension we vary is whether or not constructors are strict.  Other
dimensions are possible, for example whether \texttt{let} bindings are strict,
but we restrict ourselves to function/application and constructor strictness.
The STG machinery makes the choice of evaluation strategy easy to change,
though changing STG generation can be more efficient by avoiding the
unnecessary creation of \emph{THUNK}s as discussed in Appendix~A.


\subsection{STG to C}
The back-end is written in Haskell and generates C code; the runtime system is
written in C/C++. This gives us the extensive optimization performed by modern
C compilers such as GCC and Clang/LLVM. The system fairly faithfully
implements the STG machine as described in \emph{EA} and dynamically
illustrated by Pope's Ministg~\cite{ministg} with the exception of the state
transition rules which may be different than those for non-strict semantics.
Sharing is preserved.

\subsubsection{Proper tail calls in C.}
Parallelism is achieved by multi-threading; our model is based on Harris et
al.'s.~\cite{Harris:2005}; later developments by Marlow et al.\ are beyond our
aspirations~\cite{Marlow:2009,Marlow:2011}.  To achieve efficient parallel
execution they identified the need for the ability to examine and unwind the
stack in quashing concurrent evaluation of the same \emph{THUNK}\@.  This
requires that use of the C stack be highly constrained if we are to avoid
complex and ugly hacks using, e.g., \emph{setjmp/longjmp}, or non-portable
code to directly manipulate the C stack, complicating an already non-trivial
implementation effort.  We therefore must limit the use of the C stack to
calls to the runtime, in other words, when executing code generated from the
user program (and not in a runtime call), there must be no growing of (pushing
return addresses or data onto) the C stack.
%
To clearly distinguish what a C call (a syntactic concept) can compile to, we
will refer to a machine-level call---pushing a return address onto the native
C stack---as \emph{call-with-return}, and to a machine-level jump as
\emph{jump}.

First we establish that this is possible in principle on x86-64/Linux.  By
maintaining our own data and control (continuation) stack all generated C
functions can take a uniform (including empty) set of explicit arguments
(passed in registers) and have uniform return type. If we arrange that
generated C code has only calls (other than runtime or other external calls,
which do not call back to user-program code) in tail call positions, that is,
immediately before a return in the control flow graph, then at non-zero
optimization levels both GCC and Clang/LLVM will compile C calls to jumps
instead of calls-with-return.  When using GCC we can also use an explicit set
of machine registers for passing/returning values, much like GHC using the
LLVM back-end with its custom GHC calling convention.

The STG abstract machine is implemented using a combined data and continuation stack.
Continuations contain a code pointer, optionally some data, and layout
information for the benefit of the garbage collector.  
% For example, the argument frame for a function call is one type of
% continuation.
When the evaluation of a C function (corresponding to some STG code) is
completed its return is by an indirect jump using the code pointer in the
continuation on the top of the stack (``returning through the continuation
stack'').
%In the case of an argument frame for a function, the associated code simply
%pops the continuation off the stack and returns through the next continuation
%on the stack.

\emph{EA} describes a family of generic C-{}- \emph{stgApply} runtime
functions used for applications of non-known (not statically identifiable)
functions, and applications of known functions to other than their arity
number of arguments.  There the object to be applied, if a \emph{THUNK}, is
first evaluated with call-with-return.  When a (resulting) \emph{FUN} or
\emph{PAP} is applied to more than its arity number of arguments, it is
called-with-return with its arity number of arguments, and the resulting value
applied to the remaining arguments.
%
Call-with-return becomes even more convenient for strict evaluation: arguments
can be evaluated in a tight loop, and similarly for strict constructors, if
the STG generator (human or otherwise) has not arranged for them to be already
evaluated.
%When considering more strict semantics call-with-return is even more
%convenient because arguments can be evaluated in a tight loop.

These uses of call-with-return, which in general cannot be inlined, grow the C
stack, and without bound.  Our solution uses a single C \emph{stgApply}
function and an auxiliary function with no calls-with-return.  This is
straightforwardly realizable because the stack is explicit, and with modern C
compiler call optimization the continuation style enables a jump mechanism
where the \emph{code labels}---entities that name blocks of code, can be
treated as first class (e.g., stored for later use), and serve as targets of a
jump---are C function addresses (c.f.\ \S 6.1~\cite{PJ-stockhardware}).
%For \emph{direct calls}, in which a known function is applied a number of
%arguments equal to its arity, \emph{stgApply} need not be invoked: the
%\emph{FUN} can be tail-called directly.

\begin{comment}
\subsubsection{A taste of the continuation style.}
Now to implementation.  To clearly distinguish between a notional C call (a
syntactic concept) and a machine-level call (pushing a return address onto the
native C stack) we will refer to the latter as \emph{call-with-return}, and to
a machine-level tail call as \emph{jump}.

First some details.  Our implementation of the STG abstract machine has an
abstract register called the \emph{current value} register or
\emph{CV}.\footnote{There are actually two registers, one for pointers and
  one for unboxed values, but for simplicity we minimize discussion about
  unboxed values.}  Roughly speaking, it can hold the pointer to what is about
to be evaluated or what was just evaluated.  For example, when initiating
evaluation of a \emph{THUNK} object, \emph{CV} contains a pointer to that
object; it is much like the \emph{self} pointer in C++.  On completion of
evaluation \emph{CV} will contain a pointer to the object to which the
\emph{THUNK} evaluated.  A \emph{THUNK} will not return another
\emph{THUNK}---an evaluation is always ``big step'' to (at least) WHNF, that
is, to a \emph{FUN}, \emph{PAP}, or \emph{CON}.
% These latter, if evaluated, simply return through the continuation stack,
%effectively returning \emph{self}.

Continuations contain a code pointer, optionally some data, and layout
information.  These continuations are maintained on a stack.  When the
evaluation of an object is completed its return is by an indirect jump using
the code pointer in the continuation on the top of the stack (``returning
through the continuation stack'').  For example, the argument frame for a
function is one type of continuation and its associated code simply pops the
continuation off the stack and returns through the next continuation on the
stack.

Before evaluating a user program an initial continuation is pushed onto
the stack that points to a C function that performs whatever post-user-program
processing might be wanted, such as displaying the result of the program in
\emph{CV}\@.  The driver then jumps to the entry point of the user program,
which will ultimately return through this continuation.  This makes clear
that continuations direct \emph{where to go to}, not \emph{how to return
to the caller} as in a stack for an imperative language, despite the use of
the term ``return.''

Now consider the evaluation of \texttt{case e of x \{ alts \}} if
call-with-return is disallowed.  If the compiler cannot determine that
evaluating \texttt{e} does not jump somewhere then there must be some
mechanism to get to the \texttt{alts}.  Our solution follows \emph{EA}, using
a separate C function for the \texttt{alts}, and before executing the code to
evaluate \texttt{e}, to push a continuation---the address of this function and
the free variables of the \texttt{alts}---onto the stack.  Whatever the code
for \texttt{e} might do, ultimately (in the absence of error or
non-termination) the value of \texttt{e} is in \emph{CV}, and some code
somewhere has reached its end and returns through the continuation stack, thus
entering the code for \texttt{alts}, which immediately associates \texttt{x}
with the value in \emph{CV}.

\emph{EA} describes a family of generic C-{}- \emph{stgApply} runtime
functions used for applications of non-known (not statically identifiable)
functions, and applications of known functions to other than their arity
number of arguments.  This family is parameterized by the number of arguments
in the application and whether each is boxed or unboxed.  Because we have an
explicit stack we are able to use a single C \emph{stgApply} function taking a
contiguous block of machine words (unboxed values or pointers) on the explicit
stack.  Some modifications are required, however, because \emph{EA} uses
call-with-return.  Specifically, when a \emph{FUN} or is \emph{PAP} applied to
more than its arity number of arguments a call-with-return to the (underlying)
\emph{FUN} is made with the arity number of arguments, then the appropriate
\emph{stgApply} function is jumped to, passing the newly returned object and
the remaining arguments.  Call-with-return becomes even more convenient for
strict evaluation: arguments can be evaluated in a tight loop, and similarly
for strict constructors, if the STG generator (human or otherwise) has not
arranged for them to be already evaluated.

Our solution to this entails passing \emph{stgApply} the object to be applied
in \emph{CV}, together with an extra argument---call it
\emph{index}---indicating which of the object to be applied and its boxed
arguments have been already been evaluated (previously by \emph{stgApply}).
If the object to be applied, or some argument (because of the strictness
level), remains to be evaluated, it is jumped to, with return as ever through
the continuation stack.  This is actually fairly efficient because we can
arrange that the stack frame for \emph{stgApply} is reused, updating only
\emph{index}, so this is realized as a machine-level loop using an indirect
jump.  For the cases of arguments in excess of the arity of the object being
applied, in effect \emph{stgApply}'s argument frame is popped and two
continuations are pushed, the first one to jump to \emph{stgApply} with the
excess arguments, the next an argument frame for the object to applied, then
the code for the object to be applied is jumped to.
\end{comment}

\begin{comment}
\subsubsection{Is this too much for this paper?}
Similar to what is described in \emph{EA}, in an earlier
implementation we have a generic variadic \emph{stgApply} C function that is
used for applications of non-known functions, and known functions with a
deficit or excess number of the arguments.  The \emph{stgApply} function takes
as arguments the first element of the application (\emph{FUN}, \emph{PAP},
\emph{THUNK}, or \emph{BLACKHOLE}) and its arguments, and if a \emph{THUNK},
evaluates it with a call-with-return to yield a \emph{FUN} or \emph{PAP}.
This allows us to implement a single \emph{stgApply} function as a C function
taking a contiguous block of machine words (unboxed values or pointers) on the
explicit stack.  The implementation of \emph{stgApply} as described in
\emph{EA} performs the same call-with-return, but the scheme described there
uses an auto-generated family of \emph{stgApply} functions parameterized by
number and type (unboxed or pointer) of arguments.  (We used this scheme
initially as well, but because the stack is now fully explicit and the stack
frame is self-describing (e.g., contains a bitmap describing which arguments
are boxed/unboxed for the benefit of the garbage collector), this elaboration
is unnecessary.) Call-with-return was also used in the case that a \emph{FUN}
or \emph{PAP} is provided more than its arity number of arguments, in which
case \emph{FUN} or \emph{PAP} is called-with-return with its arity number
of objects, and the returned object and remaining argument passed (via jump)
to \emph{stgApply}.  Finally, for strict evaluation orders it is straightforward
to evaluate function or constructor arguments one at a time with call-with-return.

At that point of development these calls-with-return were clearly a hack
in an otherwise continuation-passing (via the explicit stack) control flow.
Because call-with-return could be of unbounded call depth this would be
problematic for stack unwinding.

The solution, slightly simplified, entailed adding a new parameter to \emph{stgApply}
and make the first parameter, the object to be applied, implicit.  The implicit
parameter is in \emph{CV}---a pointer to the object to be applied when jumping to \emph{stgApply}.
The new argument simply indicates which of the boxed arguments

For \emph{direct calls}, in which a known function is applied
to a number of arguments equal to its arity, \emph{stgApply} need not be
invoked: the \emph{FUN} is tail-called directly.


In addition to being in a tail call position, the GCC constraints for
generating a tail call (jump) are (approximately) that the total size of the
caller and callee arguments be equal, and the return type sizes be equal.  For
Clang the requirement is slightly more strict: caller and callee type
signatures must be the same.  These are a consequence of the C calling
convention that the caller, not callee, cleans up the stack (removes the
callee's stack frame).  However, because we maintain our own stack we can
define our own calling convention(s), so tail calls can be made to functions
of notionally differing type.  In particular, top-of-stack frames can be
adjusted in size and content without undue copying.
\end{comment}

\subsection{Type system}

Type information is required for code generation.  For example, at the
coarsest level it is necessary to know whether values are boxed, or otherwise
represented by pointers, to avoid expensive tagging.  Because we
do not yet get type information directly from GHC we use a subset of Haskell
syntax (e.g., no named fields) for defining algebraic data types, extended
with built-in and user-defined unboxed types.  We perform standard
Hindley-Milner type inference and enforcement following Heren et
al.~\cite{Heren02}, again extended with unboxed types.  The extensions to the
type inference algorithm, syntax, and restrictions for unboxed types follow
Peyton Jones and Launchbury~\cite{Jones:1991}.
%\begin{comment}
% breaks the page limit
For convenience we allow general recursive \texttt{let}s that are implicitly
factored in the usual way to maximize let-polymorphism; this would normally
already be performed by a front-end.
%\end{comment}
\begin{comment}
 breaks the page limit
One would expect an automated STG generator such as GHC to factor recursive
\texttt{let} expressions into nested recursive and non-recursive \texttt{let}s
in the usual way to maximize let-polymorphism.  Because our STG is intended
to be somewhat convenient to write by hand we relax this constraint and perform
this transformation implicitly using a strongly-connected components algorithm
inspired by King and Launchbury~\cite{King:1995}.
\end{comment}

\section{Current Status (June 8, 2016), Future Work}

An almost unbounded amount of interesting development suggests itself; one
such path could in principle lead to the recreation of GHC\@.  However, given
the minuscule budget (and therefore the employment of student interns) we must
maintain sharp focus on the main goals: implementing and evaluating mechanisms
for automatic parallelism and fault tolerance.  At the time of writing we have
a fully functioning serial implementation including
\begin{itemize}
\item STG parser \emph{(by student)},
\item mini-Haskell parser, mini-Haskell to STG transformer \emph{(by student)},
\item type inferencer,
\item various analysis and transformation passes \emph{(some by students)},
\item code generator \emph{(some student contribution)},
\item runtime system including a garbage collector,
\item continuous integration and testing system,
\item extensive test suite \emph{(mostly by students)}.\footnote{Many are
    students' solutions to exercises in learning algorithms and data
    structures.}
\end{itemize}

The \emph{mini-Haskell} front-end is detailed in Appendix~\ref{app:minihaskell}.

We use a continuous integration system augmented with logic to categorize
programs in a large and growing test suite according to which degrees of
non-strictness they require to terminate correctly, and to provide other
logging and debugging information.  This is detailed in Appendix~\ref{app:testing}.

While we can parse and type-check more general unboxed types, code generation
and runtime support is currently limited to built-in single-word (64-bit) or
smaller types (e.g., \texttt{Int\#}, \texttt{UInt\#}, \texttt{Float\#},
\texttt{Double\#}) and user-defined enumerations (simple sums), e.g.,
\texttt{data unboxed Bool\# = False\# | True\#}.
%\begin{verbatim}
%     data unboxed Bool# = False# | True# .
%\end{verbatim}

For simplicity the implementation is 64-bit only.
%
The current, primitive parallel runtime is based on Pthreads and a
non-blocking work queue~\cite{Michael:1996}.  We are currently developing a
more realistic parallel runtime, experimenting with using the Argobots
user-level thread library~\cite{Seo:2015}.
%
The fault-tolerance mechanisms remain future work.
%
Parallel garbage collection is likely out of scope.

This summer (2016) a returning student will bridge the gap between GHC STG (or
Core) and our STG, augmenting the runtime as needed.  Another student will
develop an instrumentation infrastructure integrated with the testing
framework to record and present time and space run-time statistics such as
number of heap and stack allocations and maximum stack depth.


%We have implemented a simple copying garbage collector~\cite{Cheney:1970} with
%small \emph{nurseries} in the multithreaded case~\cite{Marlow:2008}. Marlow:2011 as well
%\cite{Harris:2005,Marlow:2008,Marlow:2009}


\section{Related Work}

\emph{Nihil sub sole novum}---there is nothing new under the sun. % (Ecclesistes 1:9).
%every new idea has some sort of precedent or echo from the past~\cite{nihil}.

%Previously cited papers describing some of the inner workings of GHC at
%various points in its evolution have greatly informed our approach.

\subsubsection{Strict but still pure.}
The concept of strict but still pure in some sense dates back to the concept
of applicative order reduction in the lambda calculus.  In terms of a
notionally concrete programming language the first thorough explications may
be Backus' FP~\cite{Backus:1978} and follow-on FL
implementation~\cite{AikenFL,FLreport89}.
%
A notable argument for the potential merits of a strict-by-default
pure functional language over a non-strict one is by Sheard, who also provided
a very simple embedded implementation~\cite{Sheard:2003}.
%
A plug-in for GHC (prior to 8.0) exists that makes function bodies strict in
their non-recursive \texttt{let} bindings using a transformation on GHC Core
that recursively expands those \texttt{let} bindings into \texttt{case}
bindings~\cite{Bolingbroke:2008,strict-ghc-plugin}.
%
The Disciplined Disciple Compiler project appears to intend implementation of
a strict, higher order, functional language (or family of languages) compiler
and has been underway for some time, but the emphasis appears to be on a much
more complex type system for handling stateful computation, and the prospects
for its eventual completion are unknown~\cite{disciplined-disciple}.

\subsubsection{Strict, pure, auto-parallelizing.}
NVIDIA's NOVA language, and corresponding compiler, implements a strict
(``call by value''), pure, polymorphic, higher-order functional language with
built-in parallel operators (e.g., \emph{map}, \emph{reduce}, and
\emph{scan})~\cite{NVIDIA:2013}. Three back-ends were implemented: for
sequential C, parallel (multi-threaded) C, and CUDA C, the lattermost
targeting NVIDIA GPUs.

\subsubsection{GHC as a front-end.}
The Intel Labs Haskell Research Compiler (HRC) is the most closely related
work to ours of which we are aware~\cite{Liu:2013,Petersen:2013}.  As in our
planned approach they use GHC as the front end, but exit GHC at the final GHC
Core stage, just before transformation to STG.  The essence of their effort is
to target an existing, highly optimizing, Intel functional language compiler
that is ``largely language agnostic'' and performs (among many other
optimizations) SIMD vectorization.

Don Stewart described an implementation wherein GHC was used as a Haskell
compiler front-end, dumping STG code as text, parsing that text, then
translating to Java code for compilation and execution on the Java Virtual
Machine~\cite{Stewart-BSc}.

\subsubsection{Fault tolerance for functional programming.}
Pointon et al.\ developed a limited fault tolerance mechanism for Glasgow
distributed Haskell based on distributed exception
handling~\cite{Pointon:2001}.  However, it appears that their scheme for
handling failed processing elements correctly may have
never been implemented~\cite{Trinder:2000}.

More recently, Robert Stewart extended Haskell Distributed Parallel Haskell
(HdpH)~\cite{hdph} to be fault resilient at an impressive scale---1400 cores
on a distributed memory system.  Here the key recovery strategy is task
replication~\cite{Stewart:2013}.

\section{Acknowledgments}

Funding for this project has been provided by the DOE NNSA LANL Laboratory
Directed Research and Development program award 20150845ER; the National
Science Foundation Science, Technology, Engineering, and Mathematics Talent
Expansion Program (NSF STEP) program for undergraduate students; and the Department of Energy
Science Undergraduate Laboratory Internship (DOE SULI) program.

Los Alamos National Laboratory is managed and operated by Los Alamos National
Security, LLC (LANS), under contract number DE-AC52-06NA25396 for the
Department of Energy's National Nuclear Security Administration (NNSA).


%
% ---- Appendix ----
%

\appendix

\section{Mini-Haskell}
\label{app:minihaskell}
As mentioned, a current near-term goal is to use GHC as our front-end to
generate STG (or Core) as input to our system.  In the meantime, as a stop-gap
measure to make the generation of test cases easier, we have implemented a
\emph{mini-Haskell} front-end that supports a subset of standard Haskell
syntax with \emph{MagicHash} enabled, with a single extension.  Recall that
\emph{MagicHash} allows \texttt{\#} to be a suffix of an identifier, or denote
an unboxed literal value.  Our extension (as for STG) is the provision of
the optional \texttt{unboxed} keyword in datatype definitions to allow
user-defined unboxed data types, e.g.,
{\footnotesize
\begin{verbatim}
   data unboxed Bool# = False# | True# .
\end{verbatim}}

Writing STG in the concrete syntax conceived for our project is tedious and
prone to errors.  Compared to Haskell, the eventual source language for the
compiler, STG is relatively verbose, requiring explicit block and expression
delimiters, and classification of heap objects. Though its general structure
is similar to primitive Haskell, it lacks many of the conveniences that make
Haskell a pleasure to program in.  The motivation for a more expressive
language for testing our compiler and runtime system in their nascent stages
is then clear.  This language, mini-Haskell, was progressively ``grown'' from
the STG syntax towards a subset of Haskell itself, removing much of the
frustrating syntactic cruft and adding some of the core features that are
often taken for granted in Haskell (and other languages).  An alternative
approach would translate from mini-Haskell to GHC Core, then either to STG or,
in the case that the back end eventually take, compiling Core directly.

An informal specification of the mini-Haskell grammar is given in
Figure~\ref{fig:miniHaskell}.  There is significant overlap with STG resulting
from the structural similarities, but the redundancies are included for
completeness.


% Haskell Report:
%   https://www.haskell.org/onlinereport/haskell2010/haskell.html
% (PDF):
%   https://www.haskell.org/definition/haskell2010.pdf

% Chapter 10:
%   https://www.haskell.org/onlinereport/haskell2010/haskellch10.html
% (No standalone PDF)


% because # looks strangely large next to math terms
\newcommand{\hash}{{\scriptsize\#}}

\begin{figure}
\centering
\footnotesize % tiny scriptsize footnotesize small
\begin{tabular}{r r c l l}
Variable         & $f,\ x$             &     &  & Initial lower-case letter \\

Constructor        & $C$                 &     & & Initial upper-case letter \\

Literal            & $\mathit{lit}$      & ::= & $i\ |\ d\ 
                                                 |\ i$\hash\ $|\ d$\hash & Integral or floating point literal \\

                   &                     &     &  & \# suffix denotes unboxed value \\

Atom               & $a$                 & ::= & $\mathit{lit}\ |\ x\ |\ C$ & Constructors are first class \\% not sure if that's the right terminology 
\\


Expression         & $e$                 & ::= & $a$ & Atomic expression \\

                   &                     & $|$ & $e\ e$ & Expression application \\

                   &                     & $|$ & \texttt{\textbackslash} $\mathit{pat}_i$ 
                                                 \texttt{->} $e$ & Lambda expression, $i > 0$ \\

                   &                     & $|$ & \texttt{case} $e$ \texttt{of} $\mathit{alt}_i$ & Case expression, $i > 0$\\

                   &                     & $|$ & \texttt{let} $\mathit{odecl}_i$ \texttt{in} $e$ & Recursive let expression, $i > 0$ \\

\\
Pattern            & $\mathit{pat}$      & ::= & $C\ \mathit{pat}_i$ & Nested constructor matching \\
                   &                     & $|$ & $\mathit{lit}$ & Match numeric literals \\
                   &                     & $|$ & $x$ & Bind to variable \\

\\
Alternative        & $\mathit{alt}$      & ::= & $\mathit{pat}_i$ \texttt{->} $e$ & $i > 0$ \\
\\
Type               & $\mathit{type}$     & $|$ & $C$ & Concrete type (constructor naming) \\
                   &                     & $|$ & $x$ & polymorphic type variable \\
                   &                     & $|$ & $\mathit{type}$ \texttt{->} $\mathit{type}$ & Function type \\
\\
Object decl. & $\mathit{odecl}$    & ::= & $x$ \texttt{::} $\mathit{type}$ & Type signature \\
                   &                     & $|$ & $x = e$ & Simple binding \\
                   &                     & $|$ & $f\ \mathit{pat}_i = e$ & Function declaration \\
                   &                     &     & & $e$ cannot be an unboxed literal \\
\\
Constructor defn. & $\mathit{con}$  & ::= & $C\ \mathit{type}_i$ & $i \ge 0$ \\
\\
Datatype defn. &  $\mathit{ddecl}$ & ::= & \texttt{data} [\texttt{unboxed}] & User-defined data type  \\
               &                   &     & $C\ x_i =$                       & $i \ge 0, n > 0$         \\
               &                   &     & $\mathit{con}_1 | \dots |  \mathit{con}_n$ \\

\\
Program                & $\mathit{prog}$ & ::= & $\mathit{(o|d)decl}_i$ & Object and data defns
\end{tabular}
\caption{mini-Haskell syntax}
\label{fig:miniHaskell}
\end{figure}


Because mini-Haskell was designed to supplement STG for test generation, it
makes sense to talk about it in terms of the differences between the two and
how mini-Haskell is translated into STG for the back-end.  Here, the
differences are given in order of perceived utility, or importance in
transitioning from STG to mini-Haskell.

\subsubsection{Removal of explicit heap objects.}
In STG, all object definitions require explicit classification with
\texttt{FUN}, \texttt{PAP}, \texttt{CON}, \texttt{THUNK}. This classification can be
inferred from context by the compiler so these keywords are not necessary.
Functions can then be defined in Haskell style with the formal parameters following
the function's identifier on the left side of the equality. With these simple
changes the syntax becomes a subset of Haskell.

Heap objects are still only created at the top level or in \texttt{let}
expressions, so the translation is straightforward: \texttt{FUN} objects are
identifiable by the inclusion of these parameters.  A \texttt{PAP} object is
created when a known function is partially applied.  \texttt{CON} objects are
created when constructors are fully applied.  Every other object becomes a
\texttt{THUNK}.

\subsubsection{Avoiding building thunks.}
As \emph{EA} points out, there is no need to build a thunk for the scrutinee
of a \texttt{case} expression because if the \texttt{case} is evaluated the
scrutinee is certain to be evaluated.  Similarly, if a function is strict
there is no need to build a thunk for each of its arguments (assuming boxed
argument type).  Because STG allows only \emph{atoms} (variables or literal
values) in argument positions, naive translation of Haskell \texttt{f e} to
\texttt{let \{x = e\} in f x}, followed by naive code generation, will result in
an unnecessary heap allocation.  An alternative is to translate to
\texttt{case e of x \{\_ -> f x\}}, which we regard as a role of a front-end
such as mini-Haskell.

\subsubsection{The layout rule.}
Mini-Haskell implements Haskell's layout rule in an almost direct translation
of the specification given in section 10.3 of the Haskell 2010
Report~\cite{haskell2010report}. The notable exception is the omission of any
attempt to define the \texttt{parse-error} function.  This is apparently a
tricky spot for Haskell compilers (even GHC has not always had it right) so,
in the name of tradition, it remains a tricky spot for our mini-Haskell front
end.  The edge cases that it will incorrectly handle as a result of this are
exactly that: edge cases.  As such, they are easy to avoid in practice.

The inclusion of the layout rule may seem insignificant, but it
allows users to produce code that is visually similar to Haskell code.
This makes it easier to read and write for any Haskell programmer who relies
on layout to delineate context, presumably the vast majority.  It also allows
for the use of Haskell programming environments that support ``smart''
indentation, e.g., \emph{haskell-mode} for Emacs~\cite{haskellmode}.



% Layout Rule Spec:
%   https://www.haskell.org/onlinereport/haskell2010/haskellch10.html\#x17-17800010.3
% GHC's syntax "infelicities":
%   https://downloads.haskell.org/~ghc/latest/docs/html/users\_guide/bugs-and-infelicities.html\#infelicities-syntax
% More on parse-error:
%   http://www.dcs.gla.ac.uk/mail-www/haskell/msg01711.html


\subsubsection{Expression to expression application.}

In STG function application must be to atomic values, i.e., either literals or
variables.  This often requires introducing bindings with either \texttt{let} or
\texttt{case} expressions.  Consider the \texttt{append} function that
concatenates its two list parameters, which can be expressed in STG by
{\footnotesize
\begin{verbatim}
   append = FUN(l1 l2 ->
               case l1 of
                 { Nil -> l2;
                   Cons hd tl -> 
                     let { rec = THUNK(append tl l2);
                           result = CON(Cons hd rec) }
                     in result })
\end{verbatim}}
\noindent In mini-Haskell expressions may be applied to other expressions, so the
recursive call to \texttt{append} can be parenthesized. There is no need for the
\texttt{let} expression:
{\footnotesize
\begin{verbatim}
   append l1 l2 = case l1 of
                    Nil -> l2
                    Cons hd tl -> Cons hd (append tl l2)
\end{verbatim}}
\noindent This also serves as a good example of how expression to expression application is
transformed into STG.  Any non-atomic expression in an application expression is
bound to a heap object in a generated \texttt{let} expression.  These
expressions are thus properly ``atomized'' for STG.  The above mini-Haskell code
is transformed into STG that is structurally identical to the STG code that
precedes it.

\subsubsection{Fancy pattern matching.}

STG supports pattern matching but only in a simple form and only in
\texttt{case} expressions.  Mini-Haskell extends this to something approaching
what is found in Haskell.  Functions, including lambda expressions, may match on
all their arguments. Named functions may have multiple definitions to exhaust
the combinations of patterns in their arguments and patterns in all contexts
(functions and case expressions) may be nested arbitrarily deep.  This feature
greatly simplifies the task of defining more complex behavior.  Without it,
each matched parameter of a function requires an additional level of nested
\texttt{case} expressions. The same is true of each level of nesting within a
single pattern.
%
Even relatively simple functions can be painful to write, such as \texttt{unzip}:
{\footnotesize
\begin{verbatim}
   unzip = FUN(list ->
     case list of { Nil -> let {res = CON( TP2 nil nil)} in res;
                    Cons x xs ->
                      case x of {
                        TP2 a b ->     --TP2 is the pair constructor
                          case unzip xs of {
                            TP2 as bs ->
                              let {aas = CON(Cons a as);
                                   bbs = CON(Cons b bs);
                                   res = CON(TP2 aas bbs)}
                              in res }}})
\end{verbatim}}
\noindent In mini-Haskell this becomes
{\footnotesize
\begin{verbatim}
   unzip :: List (Tupl2 a b) -> Tupl2 (List a) (List b)
   unzip Nil = TP2 Nil Nil
   unzip (Cons (TP2 a b) xs) =
     case unzip xs of
       TP2 as bs -> TP2 (Cons a as) (Cons b bs)
\end{verbatim}}

Of course, this complex pattern matching must eventually be transformed into
the primitive, nested case expressions. Fortunately there are well-documented
algorithms for achieving this, and Philip Wadler's chapter
% in \emph{The Implementation of Functional Programming Languages}
on the compilation of this type of pattern matching proved
invaluable~\cite{Jones:1987}.  Wadler describes the \emph{match} function
which, in its simplest form, produces a correct, but potentially large
translation with some redundancy.  This function was implemented with only
minor modifications to fit the shape of mini-Haskell's abstract form.  The
implementation is augmented by some light flow-analysis to eliminate some of
the redundancy and make the resulting code easier to read.  There are further
optimizations suggested by Wadler to address the size of the generated code,
for example, when multiple branches lead to the same large expression, but the
additional complexity outweighs the benefit for a testing language.


\subsubsection{Less significant features.}

Mini-Haskell provides some smaller conveniences that don't warrant individual
sections but are worth mentioning.
\begin{itemize}
\item \emph{Constructors as functions.}
  As in Haskell, constructors are first class, acting as true functions.  This
  is implemented by simply generating a function that wraps the appropriate
  \texttt{let}-bound \texttt{CON} object in STG.  Any partially applied
  constructor generates a call to this function.  Fully applied constructors
  generate local \texttt{let}-bound objects to avoid an unnecessary function
  call.

\item \emph{Primitive operations as functions.}
  Similar to constructors, primitive operations are given first-class status as
  functions, with roughly the same rules.  Partial applications are
  translated to calls to generated functions that wrap the actual primitive
  operations. Fully applied calls generate true primitive operation calls.

\item \emph{Lambda expressions.}
  mini-Haskell supports anonymous lambda expressions for the simple, one-off
  functions that often find a place in Haskell code.  This was certainly the
  simplest feature to add, requiring only that the function be given a unique
  identifier to bind it to a \texttt{FUN} object in STG.

\item \emph{Type annotations.}  The ability to assert the type of some
  variable is often valuable as a sanity-check and provides documentation that
  the compiler can verify.  Mini-Haskell's type annotations (or signatures)
  behave like those in Haskell, being passed to the type checker as initial
  assumptions.

% I don't think I've described this very well at all... - david
\item \emph{Haskell Block.}
  Being a subset of Haskell, it may be desirable to test a mini-Haskell program
  compiled through our toolchain against the same program (and same source file)
  compiled with another Haskell compiler.  Because mini-Haskell exposes a
  different set of primitives the Haskell Block was introduced to allow the
  inclusion of the \texttt{MagicHash} extension and the aliasing Haskell's
  primitives (e.g. \texttt{(+)} and \texttt{(-)}) to the names given in
  mini-Haskell.  This block, delimited by \texttt{\{-\#-\}} strings, is ignored
  by the mini-Haskell compiler, and arbitrary ``real'' Haskell code can be
  placed therein to create the same mini-Haskell environment for full Haskell
  compilers.

\end{itemize}


\section{Test suite and testing}
\label{app:testing}
We have developed an extensive test suite which uses the CMake CTest
tool~\cite{cmake-book}. The full test suite is automatically run, and any
failures reported, after every commit to the Git revision control system. We
use a Jenkins continuous integration server to automate this build and testing
process~\cite{jenkins-book}. The Git repository is hosted externally so our
student collaborators can contribute at their leisure.

There are several hundred small STG and mini-Haskell programs in the test
suite, which is partitioned into a number of directories which control what
testing is done. Given our goal of exploring various levels of strictness,
most tests are run with all evaluation strategies (three degrees of function
strictness by two degrees of constructor strictness).  Interesting classes of
test programs are those that require less-strict or sharing semantics.  For
these programs we test not only that the test leads to the expected result
with less strict semantics, but also that the program returns a
\emph{BLACKHOLE} in the case of a stricter semantics. Students who were
Haskell/STG beginners have contributed substantially to the test suite. We
have also found that the test suite has been of use as a learning tool for
these students to discover what levels of non-strictness are required for their
programs, a benefit that we did not initially envision. Over time we have
developed extensive STG and mini-Haskell preludes of common programming
patterns to aid in the writing of test programs.

Another important class of test programs are those that should fail with a
known error, e.g., a parsing or typing error.  For these programs the test
suite checks the output against an expected error regular expression. This
class of test programs has proved valuable in development and debugging of
both the STG and mini-Haskell front-ends.

We also run each test with various levels of garbage collection enabled. This
has exposed bugs in both the code generator and runtime system.
Interestingly, in practice garbage collection can both mask and expose bugs.

The various combinations of strictness and garbage collection levels mean that
we run a quite large number of tests (currently order thousands). However,
this is fully automated via the testing and continuous integration
infrastructure.

We have also implemented a simple logging facility which can be controlled at 
both compile and run time which helps in determining why a test is failing.

\section{Student Preparation}

Student interns' backgrounds in computer programming has ranged from minimal
to some exposure to one or more mainstream imperative languages, Lisp, and
Haskell.  To bring them to a normalized level of proficiency their first
assignment has been to work through the entirety of Hutton's \emph{Programming
  in Haskell} textbook~\cite{Hutton-book}, substituting one of his articles
\emph{Higher-order Functions for Parsing}~\cite{Hutton-parsing:1992} or
\emph{Monadic Parser Combinators}~\cite{Hutton:1996} for the corresponding
material in the book.  This may be followed by writing a parser and ADT for a
progressively enriched lambda calculus, together with call-by-value,
call-by-name, and (in one case) call-by-need interpreters.  The concept of
compilation can be illustrated with an exercise in parsing infix arithmetic
expressions to an AST, and transforming that to a sequence of operations for a
simple stack machine.  From this point they have then been able to substantively
contribute to this project.

% breaks page limit
%  This learning phase can take the
%better part of a summer or one-semester internship, so we particularly
%seek return interns.


%
% ---- Bibliography ----
%

\bibliographystyle{splncs03}

\bibliography{tfp}

\end{document}
