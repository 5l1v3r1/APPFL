\documentclass{beamer}

%\usetheme[unclass]{LANL-beta}

% Use make print to create less inky slides for printing
\ifdefined\printableSlides
\usetheme[lessInk]{LANL-beta}
\else
\usetheme[blue]{LANL-beta}
\fi

\usepackage{graphicx}

\title{Automatic Parallelization and Transparent Fault Tolerance}
%\subtitle{(Project article)}
\author{Kei Davis, Dean Prichard, \\\emph{David Ringo}, Loren Anderson, \\and Jacob Marks\\ \ \\}
%\date[4/26/2016]{Some Random Meeting \\ April 26, 2016}
\date{Trends in Functional Programming, June 8-10, 2016}
\LAUR{LA-UR-16-24056}

\begin{document}

\begin{frame}
\maketitle
\end{frame}

\begin{frame}
\frametitle{Scientific Computing in our Microcosm}
\only<1-9>{\textbf{Local evolution of scientific computing}}
\begin{itemize}
  \item<2-9> Serial Fortran programs
    \\\only<2>{\ \ Simple, imperative.}
  \item<3-9> C
    \\\only<3>{\ \ Similar. More pleasant syntax (Free form source code!)}
  \item<4-9> Vendor communication libraries
    \\\only<4>{\ \  Non-portable, but parallel}
  \item<5-9> MPI everywhere (inter- and intra-node)
    \\\only<5>{\ \ De facto standard emerges to solve portability}
  \item<6-9> C++
    \\\only<6>{\ \ Object oriented goodness (and more!)}
  \item<7-9> MPI+\{Pthreads, OpenMP, OpenCL, CUDA, etc\}
    \\\only<7>{\ \ We augment MPI now with other explicit parallel models}
  \item<8-9> Parallel runtimes, e.g., Cilk Plus, Intel Threading Building Blocks,
    Stanford's Legion, etc.
    \\\only<8>{\ \ Automated scheduling for irregularly shaped problems.}
\end{itemize}
\only<9>{%
\textbf{
Currently MPI+threading model dominates.}}

\end{frame}


\begin{frame}
\frametitle{In Search of Automatic Parallelization\\
\emph{or at least automatic scheduling}}
  \begin{figure}
    \begin{center}
      \includegraphics[width=0.95\linewidth,height=1in]%
        {figures/S3Dtaskgraph.png}~\footnote{Courtesy Stanford Legion project}
\\[2.0ex]  % vertical space
      S3D task dependency, combustion chemistry calculation
    \end{center}
  \end{figure}
  \onslide<2->{%
    Simplest interesting chemistry, task graph much larger with more complex reactants.
    \vspace{0.1in}
    Schedule by hand?}
  \\\onslide<3->{%
    \textbf{This comes for free in the pure functional world}}
\end{frame}



\begin{frame}
  \frametitle{Transparent Fault Tolerance?}  

  \onslide<1->{%
    Processor clock speeds stopped growing a decade ago
    \\\vspace{0.1in}}

  \onslide<2->{%
    Solution? More processors.
    \\\vspace{0.1in}}
  
  \onslide<3->{%
    But, with more processors comes smaller Mean Time to Failure.
    \\\vspace{0.1in}}

  \onslide<4->{%
    Obviously, we can't just hope to get lucky on a months-long calculation}
\end{frame}


\begin{frame}
  \frametitle{Transparent Fault Tolerance?}
  \begin{columns}[t]
    \begin{column}{.48\linewidth}
      Checkpoint:
      \begin{enumerate}
      \item Processes synchronize at predetermined point;
      \item Stop the world;
      \item Dump global state;
      \item Resume computation.
      \end{enumerate}
    \end{column}
    \onslide<2->{%
    \begin{column}[T]{.48\linewidth}
      Restart:
      \begin{enumerate}
      \item Observe that application has hung/crashed;
      \item Identify last valid (complete) checkpoint image;
      \item Re-launch application specifying checkpoint image.
      \end{enumerate}
    \end{column}}
  \end{columns}
\vspace{0.1in}
\onslide<3->{%
Inherent problems:
\begin{itemize}
\item Idle processors burn tight energy budgets.
\item MTTF is shrinking and C/R isn't scaling.
\end{itemize}}
\onslide<4->{%
  \textbf{With referential transparency, this is a much smaller problem}}
\end{frame}


\begin{frame}
\frametitle{Pure Functional Semantics}
Now we have a `new' generation of scientific programmers, aka computational scientists, who
have some understanding of meaning and virtue of \emph{pure functional}, and even dabble
in Haskell programming.
\vspace{0.1in}

\onslide<2->{%
  They understand:\\
  \emph{In a multi-100,000-line program, do not temporarily alter the global speed-of-light
    `constant' variable.}
  \vspace{0.1in}}

\onslide<3->{%
  \emph{But}, these are not computer scientists:
  \begin{itemize}
  \item Passing functions as arguments is familiar (since early Fortran).
  \item Implicit space leaks are confusing.
  \item Non-strict evaluation is largely irrelevant.
  \item Unboxed primitives and arrays of the same (unboxed vector) are the primary data.
  \end{itemize}}
\end{frame}


\begin{frame}
\frametitle{Claimed Trends}
\textbf{Trends in functional programming}
\begin{itemize}
  \item<1-> Appreciation of the virtues of strict-by-default semantics.
    \\\only<1>{\ \ GHC Strict and StrictData for HPC}
  \item<2-> Transparent fault tolerance.
    \\\only<2>{\ \ Stewart's HdpH-RS}
\end{itemize}
\onslide<3->{\textbf{Trends in scientific computing}}
\begin{itemize}
  \item<4-> Implementation of pure functional computational concepts/components.    
  \item<4-> Recognition that checkpoint/restart is increasingly intractable and unscalable.
    \\\only<4>{%
      \ \ Legion is ``pure with respect to $X$''\\
      \ \ Also adding task-restarting for fault-tolerance}
\end{itemize}
\end{frame}


\begin{frame}
\frametitle{Non-strictness/laziness Anathema to Parallelism}
(Semantic) function strictness $\implies$ \\
\ \ \ function arguments can be evaluated early, in parallel\\
\ \ \ How can we achieve this?
\begin{itemize}
  \item<2-> Strictness analysis
    \\\only<2>{\ \ Can be inconclusive.}
  \item<3-> Bang patterns in bindings
    \\\only<3>{\ \ Prone to error. Hard to read.}
  \item<4-> Par/pseq, other specifications on expressions
    \\\only<4>{\ \ Ditto.}
  \item<5-> Speculative evaluation
    \\\only<5>{\ \ Difficult to get right without annotation.}
  \item<6-> \ldots
  \item<7-> \emph{Strict(er) default semantics}
\end{itemize}
\onslide<8->{%
Similarly constructors.
\\\vspace{0.1in}
Update:  Strict \texttt{let} bindings.
}
\end{frame}


\begin{frame}
  \frametitle{Musings}
  Suppose we had a strict(-er) pure functional language implementation than Haskell?
  \begin{itemize}
  \item<1-> Could we keep a large number of hardware threads (e.g., up to 256) busy with
    \emph{implicit} parallelism?
    \begin{itemize}
    \item With what parallel efficiency?
      \begin{itemize}
      \item With different scheduling strategies?
      \end{itemize}
    \end{itemize}
  \item<2-> How often do new functional programmers/old scientific programmers
    make essential use of non-strictness?  \emph{(very rarely)}
    \begin{itemize}
    \item Laziness? \emph{(never)}
    \end{itemize}
  \end{itemize}
\end{frame}


\begin{frame}
\frametitle{Project} 
A \emph{light-weight} implementation of a pure, higher-order, polymorphic,
functional language and runtime system with which we can experiment with
automatic parallelization strategies with varying degrees of language
strictness, and secondarily, with mechanisms for transparent fault tolerance.
\begin{enumerate}
\item<2-> Haskell to STG (or Core) via GHC \emph{(todo)};
  \\\only<2>{%
    \ \ STG and Core are higher-order polymorphic functional languages in their own right.}
\item<3-> STG to C \emph{(serial done, parallel in progress)};
  \\\only<3>{%
    \ \ Fairly faithful to STG papers}
\item<4-> Mini-Haskell front end (done);
  \\\only<4>{%
    \ \ Close to ML, syntactically}
\item<5-> Fault tolerance \emph{TBD}.
\end{enumerate}
\end{frame}


\begin{frame}
\frametitle{Reinventing the wheel?}  Why not use Manticore, MultiMLton, F\#,
multicore OCaml, SAC etc., \onslide<3->{or even GHC in toto?}
\begin{itemize}
  \item<2-> We have a small cadre of scientists already interested in Haskell/GHC;
  \item<3-> Very light weight:  easy to instrument, modify;
  \item<4-> Know exactly what is going on under the hood;
  \item<5-> There exist conduits for getting student interns up to speed inexpensively.
\end{itemize}
\end{frame}


\begin{frame}
\frametitle{Technical hurdles:  tail calling and unwinding the stack}
Following SPJ/STG our control structure is a stack of continuations.  
\\\vspace{0.1in}
Each continuation contains an address to \emph{go to next}, not \emph{go back
  to}, so we want something like \emph{goto} or \emph{tail calling}.
\\\vspace{0.1in}
SPJ identifies defines the term \emph{code label} as something that can be
\begin{itemize}
  \item used to name an arbitrary block of code;
  \item can be manipulated, i.e., stored for later use;
  \item can be used as the destination of a jump (not just call).
\end{itemize}
\end{frame}


\begin{frame}[fragile] % for verbatim
\frametitle{All the myriad ways in C}
\begin{itemize}
\item<1-> setjmp/longjmp;
\\\only<1>{\ \ Significant overhead and bookkeeping}
\item<2-> giant switch;
\\\only<2>{\ \ Ugly hack, not really linkable}
\item<3-> computed goto (gcc extension);
\\\only<3>{\ \ Limited to gcc}
\item<4-> trampoline, the simplest of which parameterless C functions return the
  address (function pointer) of where to go to next:
\begin{verbatim}
      while (1) f = (*f)();
\end{verbatim}
\only<4>{\ \ Our initial implementation used this approach}
\item<5-> implement an intermediate language, e.g., C-{}-/Cmm, where tail call is primitive;
\\\only<5>{\ \ Fun, but out of scope}
\item<6-> generate assembly code, e.g., LLVM, directly.
\\\only<6>{\ \ Ambitious. Haskell LLVM bindings are unreliable}
\end{itemize}
\end{frame}


\begin{frame}[fragile]
\frametitle{Modern compilers to the rescue:  sibling call}
Clang/LLVM and GCC implement a restricted form of tail call, the \emph{sibling}
call, i.e, jump reusing the TOS frame.  Here
\begin{verbatim}
void f() {
  ...
  g();
}
\end{verbatim}
function \texttt{g} is jumped to; there is no (implicit) return following
\texttt{g();}.  
\\\vspace{0.1in}
Critically, this also works for indirect calls to previously stored addresses.
\begin{verbatim}
    (getInfoPtr(stgCurVal.op)->entryCode)();
\end{verbatim}

\end{frame}

\begin{frame}
\frametitle{Sibling Call}
The restrictions are due to the x86(-64)/*nix/C calling
conventions---caller clean-up---and the possibility of pointer aliasing.
\\\vspace{0.1in}
Sufficient formula for recent gcc and Clang/LLVM on Linux or Mac OS X
\begin{itemize}
\item Call in leaf position of CFG;
\item Caller and callee have the same type signature
\item No addresses taken of formal params;
\item -O3.
\end{itemize}
Obviously this is not portable.
\end{frame}


\begin{frame}
\frametitle{Unwinding stacks}
In one version of GHC, Harris et al.\ identify the need to unwind thread stacks for shared
memory parallelism.
\\\vspace{0.1in}
Again, we do not want to contemplate using, e.g., setjmp/longjmp.
\\\vspace{0.1in}
Therefore, we do not want the C main stack or Pthreads stacks to grow.
\\\vspace{0.1in}
Exception:  calls to the runtime.
\\\vspace{0.1in}
Again, sibling call saves us.
\end{frame}


\begin{frame}
  \frametitle{Our Future}
  
  \begin{itemize}
  \item<1-> GHC tie-in and parallel implementation (hopefully) this summer
  \item<2-> Experimentation with non-trivial code
  \item<3-> Fault tolerance sometime thereafter
  \end{itemize}
   
\end{frame}


\begin{frame}
  \frametitle{Acknowledgments}

  \footnotesize{Funding for this project has been provided by the DOE
    NNSA LANL Laboratory Directed Research and Development program
    award 20150845ER; the National Science Foundation Science,
    Technology, Engineering, and Mathematics Talent Expansion Program
    (NSF STEP) program for undergraduate students; and the Department
    of Energy Science Undergraduate Laboratory Internship (DOE SULI)
    program.}

  \vspace{0.2in}
  
  \footnotesize{Los Alamos National Laboratory is managed and operated
    by Los Alamos National Security, LLC (LANS), under contract number
    DE-AC52-06NA25396 for the Department of Energy’s National Nuclear
    Security Administration (NNSA).}
\end{frame}


\begin{frame}
  \frametitle{Thank you}
  \textbf{Questions?}
\end{frame}

\end{document}



