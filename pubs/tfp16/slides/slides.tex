\documentclass{beamer}

%\usetheme[unclass]{LANL-beta}

% Use make print to create less inky slides for printing
\ifdefined\printableSlides
\usetheme[lessInk]{LANL-beta}
\else
\usetheme[blue]{LANL-beta}
\fi

\usepackage{graphicx}

\title{Automatic Parallelization and Transparent Fault Tolerance (project in progress)}
%\subtitle{(Project article)}
\author{Kei Davis, Dean Prichard, \\\emph{David Ringo}, Loren Anderson, \\and Jacob Marks\\ \ \\}
%\date[4/26/2016]{Some Random Meeting \\ April 26, 2016}
\date{Trends in Functional Programming, June 8-10, 2016}
\LAUR{LA-UR-16-?????}

\begin{document}

\begin{frame}
\maketitle
\end{frame}

\begin{frame}
\frametitle{Scientific Computing in our Microcosm}
Local evolution of scientific computing
\begin{itemize}
  \item Serial Fortran programs
  \item MPI everywhere (inter- and intra-node)
  \item C, C++
  \item MPI+X, X is Pthreads, OpenMP, OpenCL, CUDA, etc.
  \item Parallel runtimes, e.g., Cilk++, Intel Threading Building Blocks, Stanford's Legion, etc.
\end{itemize}
Currently MPI+threading model dominates.
\end{frame}

\begin{frame}
\frametitle{In Search of Automatic Parallelization\\
\emph{or at least automatic scheduling}}
  \begin{figure}
    \begin{center}
%      \fbox{
      \includegraphics[width=0.95\linewidth,height=1in]%
        {figures/S3Dtaskgraph.png}~\footnote{Courtesy Stanford Legion project}
%}
\\[2.0ex]  % vertical space
      S3D task dependency, combustion chemistry calculation
    \end{center}
  \end{figure}
Simplest interesting chemistry, task graph much larger with more complex reactants.
\vspace{0.1in}
Schedule by hand?
\end{frame}


\begin{frame}
  \frametitle{Transparent Fault Tolerance?}
  \begin{columns}[t]
    \begin{column}{.48\linewidth}
      Checkpoint:
      \begin{enumerate}
      \item Processes synchronize at predetermined point;
      \item Stop the world;
      \item Dump global state;
      \item Resume computation.
      \end{enumerate}
    \end{column}
    \begin{column}[T]{.48\linewidth}
      Restart:
      \begin{enumerate}
      \item Observe that application has hung/crashed;
      \item Identify last valid (complete) checkpoint image;
      \item Re-launch application specifying checkpoint image.
      \end{enumerate}
    \end{column}
  \end{columns}
\vspace{0.1in}
General claim:  ``data movement costs more energy than computation.''

Fact 1: In C/R, idle processors may cost more energy than data movement.

Fact 2:  C/R isn't scaling.
\end{frame}


\begin{frame}
\frametitle{Pure Functional Semantics}
Now we have a `new' generation of scientific programmers, aka computational scientists, who
have some understanding of meaning and virtue of \emph{pure functional}, and even dabble
in Haskell programming.
\vspace{0.1in}

\emph{In a multi-100,000-line program, do not temporarily alter the global speed-of-light
`constant' variable.}
\vspace{0.1in}

\emph{But}, these are not computer scientists:  
\begin{itemize}
\item Passing functions as arguments is familiar (since early Fortran);
\item Implicit space leaks are confusing;
\item Non-strict evaluation is largely irrelevant, strictness annotations are unintuitive.
\item Unboxed primitives and arrays of the same (unboxed vector) are the primary data.
\end{itemize}
\end{frame}

\begin{frame}
\frametitle{Claimed Trends}
Trends in functional programming
\begin{itemize}
  \item Appreciation of the virtues of strict-by-default semantics;
  \item Transparent fault tolerance.
\end{itemize}
Trends in scientific computing
\begin{itemize}
  \item Implementation of pure functional computational concepts/components.
  \item Recognition that checkpoint/restart is increasingly intractable and unscalable.
\end{itemize}
\end{frame}


\begin{frame}
\frametitle{Non-strictness/laziness Anathema to Parallelism}
(Semantic) function strictness $\implies$ \\
\ \ \ function arguments can be evaluated early, in parallel

\begin{itemize}
  \item Strictness analysis
  \item Bang patterns
  \item Par/pseq, other specifications
  \item Speculative evaluation
  \item \ldots
  \item \emph{Strict(er) default semantics}
\end{itemize}
Similarly constructors.
\\\vspace{0.1in}
Update:  Strict \texttt{let} bindings.
\end{frame}


\begin{frame}
  \frametitle{Musings}
  Suppose we had a strict(-er) pure functional language implementation than Haskell?
  \begin{itemize}
  \item Could we keep a large number of hardware threads (e.g., up to 256) busy with
    \emph{implicit} parallelism?
    \begin{itemize}
    \item With what parallel efficiency?
      \begin{itemize}
      \item With different scheduling strategies?
      \end{itemize}
    \end{itemize}
  \item How often do new functional programmers/old scientific programmers
    make essential use of non-strictness?  \emph{(very rarely)}
    \begin{itemize}
    \item Laziness? \emph{(never)}
    \end{itemize}
  \end{itemize}
\end{frame}

\begin{frame}
\frametitle{Project} 
A \emph{light-weight} implementation of a pure, higher-order, polymorphic,
functional language and runtime system with which we can experiment with
automatic parallelization strategies with varying degrees of language
strictness, and secondarily, with mechanisms for transparent fault tolerance.
\begin{enumerate}
\item Haskell to STG (or Core) via GHC \emph{(todo)};
\item STG to C \emph{(serial done, parallel in progress)};
\item Fault tolerance \emph{TBD};
\item Mini-Haskell front end (done).
\end{enumerate}
STG and Core are higher-order polymorphic functional languages in their own right.
\end{frame}


\begin{frame}
\frametitle{Reinventing the wheel?}  Why not use Manticore, MultiMLton, F\#,
multicore OCaml, etc., or even GHC in toto?
\begin{itemize}
  \item Very light weight:  easy to instrument, modify;
  \item Know exactly what is going on under the hood;
  \item There exist conduits for getting student interns up to speed inexpensively;
  \item We have a small cadre of scientists already interested in Haskell/GHC.
\end{itemize}
\end{frame}


\begin{frame}
\frametitle{Technical hurdles:  tail calling and unwinding the stack}
Following SPJ/STG our control structure is a stack of continuations.  
\\\vspace{0.1in}
Each continuation contains an address to \emph{go to next}, not \emph{go back
  to}, so we want something like \emph{goto} or \emph{tail calling}.
\\\vspace{0.1in}
SPJ identifies defines the term \emph{code label} as something that can be
\begin{itemize}
  \item used to name an arbitrary block of code;
  \item can be manipulated, i.e., stored for later use;
  \item can be used as the destination of a jump (not just call).
\end{itemize}
\end{frame}


\begin{frame}[fragile] % for verbatim
\frametitle{All the myriad ways in C}
\begin{itemize}
\item setjmp/longjmp;
\item giant switch;
\item computed goto (gcc extension);
\item trampoline, the simplest of which parameterless C functions return the
  address (function pointer) of where to go to next:
\begin{verbatim}
      while (1) f = (*f)();
\end{verbatim}
\item implement an intermediate language, e.g., C-{}-/Cmm, where tail call is primitive;
\item generate assembly code, e.g., LLVM, directly.
\end{itemize}
\end{frame}


\begin{frame}[fragile]
\frametitle{Modern compilers to the rescue:  sibling call}
Clang/LLVM and GCC implement a restricted form of tail call, the \emph{sibling}
call, i.e, jump reusing the TOS frame.  Here
\begin{verbatim}
void f() {
  ...
  g();
}
\end{verbatim}
function \texttt{g} is jumped to; there is no (implicit) return following
\texttt{g();}.  
\\\vspace{0.1in}
Critically, this also works for indirect calls to previously stored addresses.
\begin{verbatim}
    (getInfoPtr(stgCurVal.op)->entryCode)();
\end{verbatim}

\end{frame}

\begin{frame}
\frametitle{Sibling Call}
The restrictions are due to the x86(-64)/*nix/C calling
conventions---caller clean-up---and the possibility of pointer aliasing.
\\\vspace{0.1in}
Sufficient formula for recent gcc and Clang/LLVM on Linux or MacOS X
\begin{itemize}
\item Call in leaf position of CFG;
\item Caller and callee have the same type signature
\item No addresses taken of formal params;
\item -O3.
\end{itemize}
Obviously this is not portable.
\end{frame}


\begin{frame}
\frametitle{Unwinding stacks}
In one version of GHC, Harris et al.\ identify the need to unwind thread stacks for shared
memory parallelism.
\\\vspace{0.1in}
Again, we do not want to contemplate using, e.g., setjmp/longjmp.
\\\vspace{0.1in}
Therefore, we do not want the C main stack or Pthreads stacks to grow.
\\\vspace{0.1in}
Exception:  calls to the runtime.
\\\vspace{0.1in}
Again, sibling call saves us.
\end{frame}


\begin{frame}
\frametitle{Here is a simple frame}
\end{frame}


\begin{frame}
\frametitle{Here is a simple frame}
    \begin{itemize}
        \item Here is the first bullet in a standard bulleted list,
            as is customary for a presentation like this
        \begin{itemize}
            \item It includes some sub-bullets
            \item Here they are
        \end{itemize}
        \item Here is another bullet
        \item And here is another, with some math:
            \[ e^{i\theta} = \cos \theta + i \sin \theta \]
    \end{itemize}
\end{frame}


\begin{frame}
\frametitle{Example frame}
    \begin{columns}[t]
        \begin{column}{.48\linewidth}
            \begin{itemize}
                \item Here is a bulleted list that sits alongside a graphic
                \item With a second bullet item
                \item And a third
                \begin{itemize}
                   \item It can have sub-bullets too
                   \item Like this
                \end{itemize}
            \end{itemize}
        \end{column}
        \begin{column}[T]{.48\linewidth}
            \begin{figure}
                \begin{center}
                    \fbox{\includegraphics[width=0.9\linewidth]%
                        {LANL-logo-gray}}\\[2.0ex]
                    insert caption here
                \end{center}
            \end{figure}%
        \end{column}
    \end{columns}
\end{frame}


\end{document}



