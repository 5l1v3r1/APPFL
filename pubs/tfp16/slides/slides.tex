\documentclass{beamer}

%\usetheme[unclass]{LANL-beta}
\usetheme{LANL-beta}
\usepackage{graphicx}

\title{Automatic Parallelization and Transparent Fault Tolerance (project article)}
%\subtitle{(Project article)}
\author{Kei Davis, Dean Prichard, \\\emph{David Ringo}, Loren Anderson, \\and Jacob Marks\\ \ \\}
%\date[4/26/2016]{Some Random Meeting \\ April 26, 2016}
\date{Trends in Functional Programming, June 8-10, 2016}
\LAUR{LA-UR-16-?????}

\begin{document}

\begin{frame}
\maketitle
\end{frame}

\begin{frame}
\frametitle{In Search of Automatic Parallelization\\
\emph{or at least automatic scheduling}}
  \begin{figure}
    \begin{center}
%      \fbox{
      \includegraphics[width=0.95\linewidth,height=1in]%
        {figures/S3Dtaskgraph.png}~\footnote{Courtesy Stanford Legion project}
%}
\\[2.0ex]  % vertical space
      S3D task dependency, combustion chemistry calculation
    \end{center}
  \end{figure}
Simplest interesting chemistry, task graph much larger with more complex reactants.
\vspace{0.1in}
Schedule by hand?
\end{frame}


\begin{frame}
\frametitle{Scientific Computing in our Microcosm}
Local evolution of scientific computing
\begin{itemize}
  \item Serial Fortran programs
  \item MPI everywhere (inter- and intra-node)
  \item C, C++
  \item MPI+X, X is Pthreads, OpenMP, OpenCL, CUDA, etc.
  \item Parallel runtimes, e.g., Cilk++, Intel Threading Building Blocks, Stanford's Legion, etc.
\end{itemize}

We have a `new' generation of scientific programmers, aka computational scientists, who
have some understanding of meaning and virtue of \emph{pure functional}, and even dabble
in Haskell programming.
\vspace{0.1in}

\emph{In a multi-100,000-line program, do not temporarily alter the global speed-of-light
`constant' variable.}
\end{frame}


\begin{frame}
\frametitle{Non-strictness/laziness Anathema to Parallelism}
How to get around?
\begin{itemize}
  \item Strictness analysis
  \item Bang patterns
  \item Par/pseq, other specifications
  \item Speculative evaluation
  \item \ldots
  \item \emph{Strict(er) default semantics}
\end{itemize}
\end{frame}


\begin{frame}
\frametitle{Here is a simple frame}
    \begin{itemize}
        \item Here is the first bullet in a standard bulleted list,
            as is customary for a presentation like this
        \begin{itemize}
            \item It includes some sub-bullets
            \item Here they are
        \end{itemize}
        \item Here is another bullet
        \item And here is another, with some math:
            \[ e^{i\theta} = \cos \theta + i \sin \theta \]
    \end{itemize}
\end{frame}


\begin{frame}
\frametitle{In Search of Automatic Parallelization\\
\emph{or at least automatic scheduling}}
    \begin{columns}[t]
        \begin{column}{.48\linewidth}
            \begin{itemize}
                \item Here is a bulleted list that sits alongside a graphic
                \item With a second bullet item
                \item And a third
                \begin{itemize}
                   \item It can have sub-bullets too
                   \item Like this
                \end{itemize}
            \end{itemize}
        \end{column}
        \begin{column}[T]{.48\linewidth}
            \begin{figure}
                \begin{center}
                    \fbox{\includegraphics[width=0.9\linewidth]%
                        {LANL-logo-gray}}\\[2.0ex]
                    insert caption here
                \end{center}
            \end{figure}%
        \end{column}
    \end{columns}
\end{frame}


\end{document}



