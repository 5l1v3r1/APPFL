\documentclass{beamer}

%\usetheme[unclass]{LANL-beta}
\usetheme{LANL-beta}
\usepackage{graphicx}

\title{Automatic Parallelization and Transparent Fault Tolerance (project article)}
%\subtitle{(Project article)}
\author{Kei Davis, Dean Prichard, \\\emph{David Ringo}, Loren Anderson, \\and Jacob Marks\\ \ \\}
%\date[4/26/2016]{Some Random Meeting \\ April 26, 2016}
\date{Trends in Functional Programming, June 8-10, 2016}
\LAUR{LA-UR-16-?????}

\begin{document}

\begin{frame}
\maketitle
\end{frame}

\begin{frame}
\frametitle{Here is a simple frame}
    \begin{itemize}
        \item Here is the first bullet in a standard bulleted list,
            as is customary for a presentation like this
        \begin{itemize}
            \item It includes some sub-bullets
            \item Here they are
        \end{itemize}
        \item Here is another bullet
        \item And here is another, with some math:
            \[ e^{i\theta} = \cos \theta + i \sin \theta \]
    \end{itemize}
\end{frame}


\begin{frame}
\frametitle{Here is a more complicated frame, including some graphics}
    \begin{columns}[t]
        \begin{column}{.48\linewidth}
            \begin{itemize}
                \item Here is a bulleted list that sits alongside a graphic
                \item With a second bullet item
                \item And a third
                \begin{itemize}
                   \item It can have sub-bullets too
                   \item Like this
                \end{itemize}
            \end{itemize}
        \end{column}
        \begin{column}[T]{.48\linewidth}
            \begin{figure}
                \begin{center}
                    \fbox{\includegraphics[width=0.9\linewidth]%
                        {LANL-logo-gray}}\\[2.0ex]
                    insert caption here
                \end{center}
            \end{figure}%
        \end{column}
    \end{columns}
\end{frame}


\end{document}



