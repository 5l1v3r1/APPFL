% This is LLNCS.DEM the demonstration file of
% the LaTeX macro package from Springer-Verlag
% for Lecture Notes in Computer Science,
% version 2.4 for LaTeX2e as of 16. April 2010
%
\documentclass{llncs}
%
\usepackage{makeidx}  % allows for indexgeneration
%
% mkd for preliminary paper notes
\usepackage{enumitem} 
\usepackage{color} % for parenthetical notes while paper is in progress
\usepackage{comment} % comment environment
\usepackage{hyperref} 
%
\begin{document}
%
%mkd \frontmatter          % for the preliminaries
%
\pagestyle{headings}  % switches on printing of running heads
%
%mkd \mainmatter              % start of the contributions
%
\title{Automatic Parallelization and Transparent Fault Tolerance\\
       (New project paper)}
%
\titlerunning{Automatic Parallelization}  % abbreviated title (for running head)
%                                     also used for the TOC unless
%                                     \toctitle is used
%
\author{Kei Davis\inst{1} \and
        Dean Prichard\inst{1} \and
        David Ringo\inst{1,2} \and
        Loren Anderson\inst{1,3} \and
        Jacob Marks\inst{1,4}\thanks{All project contributors, past and present, are listed.}
}
%
\authorrunning{Kei Davis et al.} % abbreviated author list (for running head)
%
%%%% list of authors for the TOC (use if author list has to be modified)
%\tocauthor{Ivar Ekeland, Roger Temam, Jeffrey Dean, David Grove,
%Craig Chambers, Kim B. Bruce, and Elisa Bertino}
%
\institute{Los Alamos National Laboratory, Los Alamos, NM, USA\\
\email{kei.davis@lanl.gov},\\ WWW home page:
\texttt{http://ccsweb.lanl.gov/\homedir kei/}
\and
University of New Mexico,
Albuquerque, NM, USA
\and
North Dakota State University,
Fargo, North Dakota, USA
\and
New Mexico Institute of Mining and Technology,
Socorro, New Mexico, USA
}

\maketitle              % typeset the title of the contribution

% moronic author listing code increments footnote counter
\setcounter{footnote}{0}

\begin{abstract}
A claimed trend is an increasing appreciation and practice of functional
programming in scientific computing, and in particular strict (by default)
functional programming, in various guises.  A second claimed trend in both
scientific computing and functional programming is transparent fault
tolerance.  Our project goal is the demonstration of a light-weight,
higher-order, polymorphic, pure functional language implementation in which we
can experiment with automatic parallelization strategies, varying degrees of
strictness, and mechanisms for transparent fault tolerance.
%We do not consider speculative evaluation, or semantic strictness inferred by
%program analysis, so potential parallelism is dictated by the degree of
%strictness.
\\
\emph{Draft paper notes:}
\begin{enumerate}[before=\itshape]
\item We are reporting work in progress.  At the time of writing we have a
  complete serial implementation, a primitive proof-of-concept parallel
  implementation, and are currently developing a more realistic parallel
  runtime.  We expect to have preliminary results from our new parallel
  implementation by the time of the symposium.  The fault-tolerance mechanisms
  remain future work.

\item While this does not qualify as a student paper, students
  (undergraduate!)  have contributed considerably to this effort, and to the
  writing of this paper, and will continue to do so.  If this paper is
  accepted we plan to have one of these students present it at TiFP.

\item Hopefully in the spirit of a student-friendly symposium, for this draft
  version we have chosen a more informal, conversational style of presentation
  than necessarily appropriate for a final paper, have highlighted student
  involvement and its influence on our approach, and pointed out student
  contributions in particular.
\end{enumerate}


\keywords{Strict pure functional programming, automatic parallelization,
  Haskell, STG, GHC, GHC Core, transparent fault tolerance.}

\end{abstract}

\section{Background and Motivation}
The practice of so-called high-performance scientific computing (HPC) is,
overall, highly conservative with respect to change.  Even as the USA and
other governments commit to a push for exascale computing ($10^{18}$ FLOPS in
a single system) within the next
decade~\cite{Exascale-roadmap,EUexascale,Exascale-org} a highly respected
member of the HPC community recently stated publicly that ``Fortran is
essential for exascale programming.''~\cite{Heroux16} Existing US government
laboratory HPC codes run to hundreds of thousands of lines of Fortran each and
complete rewrites are infeasible.\footnote{The codes of our lab's British
  counterpart are similar.}  C++ has become more common for new code starts,
but their nature remains the same: huge codes with fair to non-existent
encapsulation of side effects, with multiple levels of parallelism sometimes
poorly abstracted, especially at the thread (shared-memory) and computational
accelerator (e.g., GPU) levels.  Thread-level parallel efficiency can be good
when the parallel model is bulk synchronous; tasking models are generally
avoided unless abstracted by, e.g., a C++ template or runtime library, because
of the programming complexity.

\subsection{A trend: strict but still pure functional programming in HPC}
Interest in functional programming for HPC in general is not new as evidenced
by the Workshop on Functional High Performance Computing, held in conjunction
with the International Conference on Functional Programming, being now in its fifth
consecutive year.  A more precise characterization of the trend is that there
is a small but growing understanding by practicing computational scientists
that \emph{pure functional semantics}, in some form, is essential to reining
in the complexity of ever-evolving scientific codes and informing the
designs of new ones.  Such semantics are expressed in a number of ways as
highlighted following.

\subsubsection{Rely on programmer discipline.}
Legacy Fortran scientific codes likely represent worst practices in state
encapsulation, with most data global and accessible to all; indeed, Fortran
was designed to make this easy via \texttt{COMMON} blocks---aggregates of
global variables---which further obfuscate program meaning by allowing
arbitrary naming of a block's variables at a subroutine/function granularity,
and requiring that entire blocks be brought into scope.  It is almost ironic,
then, that of the mainstream high-performance scientific programming languages
(Fortran, C, and C++), Fortran~95 was the first to introduce the \texttt{pure}
function qualifier, and with enforcement by the compiler.  Otherwise it is
entirely up to the programmer to make functions composable and thread safe.
%Here at LANL the use of pure functions (among other techniques for reducing
%global data dependencies) is being used to good effect to clean up some legacy
%code bases.

\subsubsection{Obey constraints prescribed by a parallel runtime system.}
There are a number of parallel runtime systems that encourage (but cannot
enforce) a pure functional style.  We highlight Stanford's Legion
runtime as an example~\cite{Bauer12}.

Using Legion, all non-function-local variables, or \emph{regions}, must be
requested from the runtime system.  Regions may be shared among \emph{tasks}
(essentially C++ functions), and each task is prescribed access privileges
(read, write, read/write, etc.) and coherency requirements for each region to
which it will have access.  Other than registered access to regions, tasks
must not access non-constant global data.  Thus the inputs and outputs of
tasks are exactly known to the runtime system.  In serial execution these access
requirements are notionally superfluous, but in parallel execution the runtime
can dynamically calculate the data dependency graph and relax the
programmatically defined serial ordering of task execution.  Very high
parallel efficiency has been demonstrated on extremely large computing
systems~\cite{Bauer14}.\footnote{It seems
  significant that the PI of the Legion project was previously a co-developer
  of a strict, arguably pure functional language
  implementation~\cite{AikenFL,FLreport89}, in collaboration with John Backus
  as a refinement of his FP language~\cite{Backus:1978}.}

\subsubsection{Use a pure functional language.}
Some twenty years after the US Department of Energy laboratory complex was
presented with the most highly parallel (and thereby performant)
implementation of a pure functional language to date~\cite{Davis96}, and
collectively had no conceptual understanding of what it was, much less why it
might be of some import, there is now a small number of laboratory HPC
practitioners who are developing an appreciation for pure functional
programming in Haskell.  Still, they see their needs as different than other
Haskell users:  performance is paramount, vectors (``Efficient
Arrays''~\cite{vector}) are a (perhaps \emph{the}) primary data structure, and
non-strictness or laziness is more of a nuisance---even a serious
hindrance---than a convenience, much less ever algorithmically essential in
the sense of Bird's \emph{repmin}~\cite{Bird84}, Johnsson's general attribute
grammars~\cite{Johnsson87}, or other instances of \emph{tying the
  knot}~\cite{tying-the-knot}.

A common complaint among these aspiring high-performance functional
programmers is the need to get strictness just right: to remove space leaks,
achieve acceptable performance, get I/O and inter-process communication to work
as they expect, and so on.  More specifically, they expect to be able to
reason about space and time usage, and order of evaluation, using their
existing mental models. While they are happy to write code that is strongly
reminiscent of the numerical examples in Hughes' classic piece (whether
they've read it or not)~\cite{Hughes89}, they would also be happy to write in
a slightly modified style that does not require non-strictness---after all,
that's how they've been thinking their entire careers.

Much anecdotal evidence suggests that many others have similar difficulties
with semantics that are lazy (or non-strict) by default, with the most telling
recent concrete evidence being the push for, and realization of, the
\texttt{Strict} and \texttt{StrictData} language extensions newly available in
GHC 8.0.  Quoting the documentation:
\begin{quote}
  High-performance Haskell code (e.g.\ numeric code) can sometimes be littered
  with bang patterns, making it harder to read. The reason is that laziness
  isn't the right default in this particular code, but the programmer has no
  way to say that except by repeatedly adding bang
  patterns~\cite{strict-strictdata}.
\end{quote}
It is a fact that \emph{HPC practitioners are generally not computer
  scientists}, they are domain scientists (physicists, chemists, etc.) or
computational scientists specialized in numerical or numerically-oriented
algorithmic methods (e.g., Lagrangian, Eulerian, adaptive mesh refinement).
The concept of a space leak, for example, caused by other than a missing
explicit deallocation statement (C \emph{free}, etc.) is difficult to
understand and not something they particularly want to try to understand.
%Hence, we eschew explicit strictness annotations for varying uniform degrees
%of strictness.

More generally, \emph{task-based} parallelism is becoming increasingly popular
for HPC, often in conjunction with a higher-level (typically inter-node) bulk
synchronous model usually implemented with the Message Passing Interface
(MPI)~\cite{MPI}\@.  In the imperative world, example systems include the
aforementioned Legion, Intel's Threading Building Blocks~\cite{Reinders:2007},
and the HPX C++11/14 extension~\cite{HPX}.  While the concept of task is
somewhat vague, the general idea is a unit of computation with no or
minimal/constrained side effects.  A function in a pure functional language
could be regarded as the essence of the concept of a task.  All of these
systems are of course in languages with strict function-call semantics.

\subsection{Another trend: transparent fault tolerance}
In the quest for computing platforms approaching exascale, it is widely
recognized that new mechanisms for fault tolerance will need to be built into
the software stack.  Current practice in HPC is checkpoint/restart, wherein at
various points in time sufficient program state is dumped to backing store
such that the program, in case of a fault, can be (manually) restarted from
the last stored state.  Most implementations are explicit, though there has
been some success with transparent (invisible to the programmer) systems.
Wiring checkpoint/restart into a large-scale application is not only burdensome and
error-prone, it can severely warp the engineering of the application
because it must be designed such that essential state can be readily captured.
Furthermore, there is a concern that the scalability of checkpoint/restart is
reaching its end.

In the imperative world side-effect-free tasks could be safely restarted after
failure, needing only their well-defined inputs and not arbitrary access to
mutable global state.  The Legion system is being augmented with this
capability, and in such a way that it will be largely transparent to the
programmer.  Ericsson's quasi-functional language Erlang and its runtime
system were designed for fault tolerance~\cite{Cesarini:2009}.  It has long
been recognized that the pure parts of a functional program (excluding, e.g.,
the I/O monad in Haskell) could be safely restarted.  The recent work of
Stewart impressively demonstrates this, in a pure functional setting, on a
large-scale distributed-memory system~\cite{Stewart:2013}.

We claim, then, that there is a trend, in HPC at least, towards
\emph{strict-by-default pure functional programming}, whether as imposed by
programmer discipline, a parallel runtime system, or the language itself
(e.g.\ strict Haskell).  Second, we claim that there is a trend in
high-performance computing, and by transitivity (in addition to Stewart's
work) in functional programming, towards transparent fault tolerance.

\subsection{Strictness and Parallelism}
It is a trivial observation that given strict functional semantics, wherein
function arguments may be safely evaluated before function evaluation, it is
safe to evaluate the arguments, and the function, in parallel, and that this
principle applies recursively in the (dynamic) expression tree (or graph).

\section{Project and Goals}
Our project is the light-weight implementation of a pure, higher-order,
polymorphic, functional language and runtime system with which we can
experiment with automatic parallelization strategies with varying degrees of
language strictness, and secondarily, with mechanisms for transparent fault
tolerance.  Light-weight has several implications.  First and foremost it
means that \emph{we are not attempting to compete with GHC in any
  respect}---succumbing to the temptation of recreating GHC language, type
system, and runtime features, or significant code optimization, would simply
divert us from our central goals.\footnote{We estimate that we will have, to
  within an order of magnitude, one person-hour of total available development
  time for each person-year that GHC currently represents, or put another way,
  one second for each hour, respectively.}  For another, it means that we are
not undertaking significant morphing of GHC itself: because this is in large
part a student project it must be kept as simple as reasonably possible in
various respects.  Our initial requirements are that
%
\begin{itemize}
\item The implementation must be feasible in terms of available effort;
\item The language must be strongly typed, implicitly or explicitly.  More
  specifically, the language should explicitly support both Hindley-Milner
  polymorphism and unboxed types~\cite{Jones:1991};
\item Straightforward direct linkage with C, preferably without a sophisticated
  foreign function interface, is essential;
  \item The language should be \emph{currying-friendly} as described by Marlow
and Peyton Jones~\cite{Marlow:2004};
\item Choice of degree of strictness should be easily selected;
\item Sharing, in the sense of what \emph{lazy} means vs.\ merely non-strict,
  should be preserved for all degrees of strictness;
\item Tail calling must be properly and consistently implemented;
\item The generated code is C.
\end{itemize}
%
For feasibility, we restrict ourselves to shared-memory implicit parallelism.
The requirement for tail calling is not primarily for efficiency, rather it
is for a reasonable parallel implementation.  Achieving this, and thereby
sufficiently limiting the use of the native C stack, makes this an interesting
exercise that we will discuss at some length.

A minor, informal goal is to gain a sense of how often relatively
inexperienced functional programmers make essential use of
non-strictness.\footnote{The starting experience of our student interns to
  date has been normalized by having them complete Hutton's \emph{Programming
    in Haskell} textbook~\cite{Hutton-book} as baseline preparation,
  substituting his article \emph{Higher-order Functions for
    Parsing}~\cite{Hutton-parsing:1992} for the corresponding material in the
  book.}

\section{Design and Implementation}
Our wish list has much in common with the GHC Haskell implementation, and
indeed GHC is part of our master plan.  In brief, the phases of the GHC
compiler are 1) parsing and type checking; 2) transformation to the GHC
\emph{Core} intermediate representation~\cite{Sulzmann:2007,ghc-core}; 3)
analysis and transformation on Core; 4) transformation of Core to the
\emph{STG} intermediate representation~\cite{PJ-stockhardware}; 5)
transformation of STG to machine code via various imperative
representations (Cmm, LLVM, C, native code generation).

In short, our plan is to use GHC as our front-end to generate one of its
intermediate forms, STG or Core, then escape to our system from there.  There
are various possible ways to accomplish the this: using GHC's option to dump
STG or Core as text, hacking GHC itself and, optionally, using GHC as a
library~\cite{ghc-as-library}.  For now we start with our own STG, with
bridging the gap with GHC scheduled to commence quite soon.

\subsection{STG Language}

Our incarnation of STG, as shown in Figure~\ref{fig:STGsyntax}, is meant to
evoke both an abstract and concrete syntax.  This presentation is adapted from
Marlow and Peyton Jones' \emph{Eval/Apply} paper~\cite{Marlow:2004}, to which
we refer so many times that we will subsequently refer to it as simply
\emph{EA}. Partial application (\emph{PAP}) and black hole (\emph{BLACKHOLE}),
i.e., a thunk being evaluated, are only runtime entities and not actually part
of our STG language.



\setlength{\tabcolsep}{5pt}
\begin{figure}
%\centering
\begin{centering}
\footnotesize % tiny scriptsize footnotesize small
\begin{tabular}{r r c l l}

Variable     & $f,\ x$        &     &                                              & Initial lower-case letter \\
Constructor  & $C$            &     &                                              & Initial upper-case letter \\
Literal      & $\mathit{lit}$ & ::= & $i\ |\ d$                                    & Integral or floating point literal \\
Atom         & $a$            & ::= & $\mathit{lit}\ |\ x$                         & \\ % Function, constructor args are atoms \\
\\
Expression   & $e$            & ::= & $a$                                          & Atom \\
             &                & $|$ & $f\ a_1\dots a_n$                            & Application, $n\ge 1$ \\
             &                & $|$ & $\oplus\ a_1\dots a_n$                       & Saturated primitive operation, $n\ge 1$ \\

             &                & $|$ & \texttt{let \{}                              & Recursive let, $n\ge 1$ \\
             &                &     & \texttt{ } $x_1$ \texttt{=} $\mathit{obj}_1$ \texttt{;} \\
             &                &     & \hspace{0.2in} $\dots$ \texttt{;} \\
             &                &     & \texttt{ } $x_n$ \texttt{=} $\mathit{obj}_n$ \\
             &                &     & \texttt{\} in} $e$  \\

             &                & $|$ & \texttt{case} $e$ \texttt{of} $x$ \texttt{\{}  & Case expression, $n \ge 1$\\
             &                &     & \texttt{ } $\mathit{alt}_1$ \texttt{;} \\
             &                &     & \hspace{0.2in} $\dots$ \texttt{;} \\
             &                &     & \texttt{ } $\mathit{alt}_n$ \\
             &                &     & \texttt{\}} \\
\\
Alternatives & $\mathit{alt}$ & ::= & $C\ x_1\dots x_n$ \texttt{->} $e$            & Pattern match, $n \ge 0$ \\
             &                & $|$ & \texttt{\_ ->} $e$                           & Default \\
\\
Heap objects & $\mathit{obj}$ & ::= &\texttt{FUN} $x_1\dots x_n$ \texttt{->} $e$   & Function definition, arity $=n\ge 1$ \\
             &                & $|$ &\texttt{CON} $C\ a_1\dots a_n$                & Saturated constructor, $n \ge 0$ \\
             &                & $|$ &\texttt{THUNK} $e$                            & Thunk---explicit deferred evaluation \\
             &                & $|$ & $\mathit{PAP}\ f\ a_1\dots a_n$               & Evaluation-time partial application \\
             &                & $|$ & $\mathit{BLACKHOLE}$                         & Evaluation-time black hole \\
\\
Program      & $\mathit{prog}$& ::= & $f_1\ =\ \mathit{obj}_1$ \texttt{;}          & $n \ge 1$, distinguished \texttt{main}\\
             &                &     & \texttt{ } $\dots$ \texttt{;} \\
             &                &     & $f_n\ =\ \mathit{obj}_n$

\end{tabular}
\end{centering}
\caption{STG syntax}
\label{fig:STGsyntax}
\end{figure}

The semantics of STG is what one would expect for a higher-order, polymorphic,
pure functional language, with a few exceptions for a Haskell programmer
familiar with unboxed types.  First, \texttt{case} is strict in the scrutinee,
so \texttt{case} is similar to \texttt{pseq} (as opposed to \texttt{seq}) in
Haskell, that is, the scrutinee is evaluated first.  Second, constructors are
not functions, rather they are named in the construction (heap allocation) of
a data object.  More generally, the operational notion of allocating a heap
object is made explicit: this is the ``side effect'' of \texttt{let} bindings.
As such a variable of unboxed type cannot be let-bound; instead it can be
bound with \texttt{case}, e.g., \texttt{case e of x\# \{ \_ -> e \}}
%\begin{verbatim}
%    case e of x# {
%      _ -> e
%    }
%\end{verbatim}
where the \texttt{\#} suffix is merely a convention to signify that
\texttt{x\#} has unboxed type.  Finally, deferred evaluation is made explicit:
a deferred expression evaluation is encoded by a thunk.

\subsection{Degrees of strictness}
First some words about terminology.  For all degrees of strictness, sharing
(as implied by lazy but not by non-strict) of heap-allocated objects is
preserved so we will not use the term \emph{lazy} and simply refer to
strictness properties.  As is common we will use strict in an operational
sense, meaning that argument(s) are evaluated before being passed in a
function call, rather than the denotational sense, unless stated otherwise.
Similarly, when referring to language semantics we refer by default to the
operational rather than denotational semantics.

Greater strictness implies more opportunities for parallelism, possible losses
or gains in space efficiency, and less expressiveness.  We seek to explore
the interplay of all of these.

GHC 8.0 gives three possibilities: conventional non-strict, constructor
strict, and function- and constructor strict, but there are possible
variations.  The obvious variation is function, but not constructor,
strictness.  In Haskell the distinction between constructors and functions is
somewhat blurred because constructors behave like functions (outside of
pattern matching), but in STG the distinction is clear.

Other possible variations concern the notional arity of the underlying
function being applied.  This is readily explained in terms of the operation
of the STG abstract machine.  Our partial description is based on that given
in \emph{EA}.

In STG the arity of a function is syntactically defined as the number of
manifest formal parameters in a function definition, thus \texttt{id x = x}
has by definition arity one regardless of the type of the argument (e.g., some
function type) it might be applied to.  We distinguish \emph{application},
e.g., \texttt{f x y}, which is a syntactic notion, from actually calling a
function. Thus \texttt{id id x} is an application of \texttt{id} to two
arguments, but the underlying \emph{FUN} (user- or system- defined function)
when actually called is given exactly one argument (the first one).

In the eval-apply model, given application \texttt{f x y}, in the standard
(non-strict) semantics \texttt{f} is first evaluated and the arity of the
underlying \emph{FUN} or \emph{PAP} (partial application of a \emph{FUN}) is
determined---in the case of \emph{PAP} the arity is that of the underlying
\emph{FUN} less the number of arguments previously provided.  If the arity is
greater than two a new \emph{PAP} is created with arguments \texttt{x} and
\texttt{y} inserted and the new \emph{PAP} returned.  If the arity is equal to
two the underlying \emph{FUN} is tail-called (jumped to) with any previous
arguments (in case of \emph{PAP}) and new arguments \texttt{x} and \texttt{y}.
If the arity is less than two (it must be at least one, so one in this case)
the underlying \emph{FUN} is called with any previous arguments (in case of
\emph{PAP}) and new argument \texttt{x}, then the object returned applied to
remaining argument \texttt{y}.

Given application \texttt{f x y} when might \texttt{x} and \texttt{y} be
evaluated?  In the non-strict semantics evaluation is on demand, after
\texttt{f} has been evaluated, the underlying \emph{FUN} actually called
(i.e., not just partially applied and wrapped up in a \emph{PAP}), and one or
both of those arguments possibly forced in subsequent evaluation (assuming
that they weren't forced in the evaluation of \texttt{f} itself).

In a strict semantics one could devise various evaluation orders, but as far
as distinguishing termination properties it seems these can be distilled down
to two reasonable possibilities, namely whether on construction of a
\emph{PAP} the (new) arguments are forced or not.  In other words, the
question is whether application is strict, or actual function call is strict.
These are in fact distinguishable because (in Haskell) the first argument to
\texttt{seq} might evaluate to a \emph{PAP}.

This then gives three reasonable operational modes that are strictly ordered
in terms of termination properties: non-strict; function strict, in which only
those arguments needed by a function call are evaluated; and application
strict, in which all arguments of an application are evaluated.  The other
dimension we vary is whether or not constructors are strict.  Other dimensions
are possible, for example whether \texttt{let} bindings are strict, but we
restrict ourselves to function/application and constructor strictness.  The
STG machinery makes the choice of evaluation strategy easy to change.
%, both in terms of function application and building a constructor in the
%heap.

\subsection{STG to C}
The back-end is written in Haskell and generates C code; the runtime system is
written in C/C++. This gives us the extensive optimization performed by modern
C compilers such as gcc and Clang/LLVM. The system fairly faithfully
implements the STG machine as described in \emph{EA} and dynamically
illustrated by Pope's Ministg~\cite{ministg} with the exception of the state
transition rules which may be different than those for non-strict semantics.
Sharing is preserved.

\subsubsection{Proper tail calls in C}
Parallelism is achieved by multi-threading; our model is based on Harris et
al.'s.~\cite{Harris:2005}; later developments by Marlow et al.\ are beyond our
aspirations~\cite{Marlow:2009,Marlow:2011}.  To achieve efficient parallel
execution they identified the need for the ability to examine and unwind the
stack in quashing concurrent evaluation of the same \emph{THUNK}.  This
requires that use of the C stack be highly constrained if we are to avoid
complex and ugly hacks using, e.g., \emph{setjmp/longjmp}, or non-portable
code to directly manipulate the C stack, complicating an already non-trivial
implementation effort.  We therefore must limit the use of the C stack to
calls to the runtime, in other words, when executing code generated from the
user program (and not in a runtime call), there must be no growing of (pushing
return addresses or data onto) the C stack.
%
To clearly distinguish between a notional C call (a syntactic concept) and a
machine-level call (pushing a return address onto the native C stack) we will
refer to the latter as \emph{call-with-return}, and to a machine-level tail
call as \emph{jump}.

First we establish that this is possible in principle.  By maintaining our own
data and control (continuation) stack all generated C functions can take no
explicit arguments and have return type \emph{void}. If we arrange that
generated C code has only calls (other than runtime calls, which do not call
back to user-program code) in tail call positions, that is, immediately before
a return in the control flow graph, then at non-zero optimization levels both
gcc and Clang/LLVM will generate jumps instead of calls-with-return for C
calls.  When using gcc we can also use a fixed set of machine registers for
passing/returning values, much like GHC using the LLVM back-end with its
custom GHC calling convention.


The STG abstract machine is implemented using a continuation stack.
Continuations contain a code pointer, optionally some data, and layout
information for the benefit of the garbage collector.  For example, the
argument frame for a function call is one type of continuation.  When the
evaluation of a C function (corresponding to some STG code) is completed its
return is by jumping through the code pointer of the continuation on the top
of the stack (``returning through the continuation stack'').  In the case of
an argument frame for a function, the associated code simply pops the
continuation off the stack and returns through the next continuation on the
stack.

\emph{EA} describes a family of generic C-{}- \emph{stgApply} runtime
functions used for applications of non-known (not statically identifiable)
functions, and applications of known functions to other than their arity
number of arguments.  Here the object to be applied is first evaluated with
call-with-return.  When a resulting \emph{FUN} is applied to more than its
arity number of arguments, it is called-with-return with its arity number of
arguments, and the resulting value applied to the remaining arguments.  When
considering more strict semantics call-with-return is even more convenient
because arguments can be evaluated in a tight loop.  Unfortunately these uses
of call-with-return can result in the C stack growing unboundedly.  Our
solution uses a single C \emph{stgApply} function with no calls-with-return.
This is straightforwardly realizable because the stack is explicit, and with
the C compiler call optimization, the continuation style provides a proper
\emph{goto} mechanism where the \emph{code labels}---entities that name blocks
of code, can be treated as first class (e.g.\ stored), and serve as targets of
a jump---are C function addresses (c.f.\ \S 6.1~\cite{PJ-stockhardware}).

\begin{comment}
\subsubsection{A taste of the continuation style.}
Now to implementation.  To clearly distinguish between a notional C call (a
syntactic concept) and a machine-level call (pushing a return address onto the
native C stack) we will refer to the latter as \emph{call-with-return}, and to
a machine-level tail call as \emph{jump}.

First some details.  Our implementation of the STG abstract machine has an
abstract register called the \emph{current value} register or
\emph{CV}.\footnote{There are actually two registers, one for pointers and
  one for unboxed values, but for simplicity we minimize discussion about
  unboxed values.}  Roughly speaking, it can hold the pointer to what is about
to be evaluated or what was just evaluated.  For example, when initiating
evaluation of a \emph{THUNK} object, \emph{CV} contains a pointer to that
object; it is much like the \emph{self} pointer in C++.  On completion of
evaluation \emph{CV} will contain a pointer to the object to which the
\emph{THUNK} evaluated.  A \emph{THUNK} will not return another
\emph{THUNK}---an evaluation is always ``big step'' to (at least) WHNF, that
is, to a \emph{FUN}, \emph{PAP}, or \emph{CON}.
% These latter, if evaluated, simply return through the continuation stack,
%effectively returning \emph{self}.

Continuations contain a code pointer, optionally some data, and layout
information.  These continuations are maintained on a stack.  When the
evaluation of an object is completed its return is by jumping through the code
pointer of the continuation on the top of the stack (``returning through the
continuation stack'').  For example, the argument frame for a function is one
type of continuation and its associated code simply pops the continuation off
the stack and returns through the next continuation on the stack.

Before evaluating a user program an initial continuation is pushed onto
the stack that points to a C function that performs whatever post-user-program
processing might be wanted, such as displaying the result of the program in
\emph{CV}\@.  The driver then jumps to the entry point of the user program,
which will ultimately return through this continuation.  This makes clear
that continuations direct \emph{where to go to}, not \emph{how to return
to the caller} as in a stack for an imperative language, despite the use of
the term ``return.''

Now consider the evaluation of \texttt{case e of x \{ alts \}} if
call-with-return is disallowed.  If the compiler cannot determine that
evaluating \texttt{e} does not jump somewhere then there must be some
mechanism to get to the \texttt{alts}.  Our solution follows \emph{EA}, using
a separate C function for the \texttt{alts}, and before executing the code to
evaluate \texttt{e}, to push a continuation---the address of this function and
the free variables of the \texttt{alts}---onto the stack.  Whatever the code
for \texttt{e} might do, ultimately (in the absence of error or
non-termination) the value of \texttt{e} is in \emph{CV}, and some code
somewhere has reached its end and returns through the continuation stack, thus
entering the code for \texttt{alts}, which immediately associates \texttt{x}
with the value in \emph{CV}.

\emph{EA} describes a family of generic C-{}- \emph{stgApply} runtime
functions used for applications of non-known (not statically identifiable)
functions, and applications of known functions to other than their arity
number of arguments.  This family is parameterized by the number of actual
arguments in the application and whether each is boxed or unboxed.  Because we
have an explicit stack we are able to use a single C \emph{stgApply} function
taking a contiguous block of machine words (unboxed values or pointers) on the
explicit stack.  Some modifications are required, however, because \emph{EA}
uses call-with-return.  Specifically, when a \emph{FUN} or is \emph{PAP}
applied to more than its arity number of arguments a call-with-return to the
(underlying) \emph{FUN} is made with the arity number of arguments, then the
appropriate \emph{stgApply} function is jumped to, passing the newly returned
object and the remaining arguments.  Call-with-return becomes even more
convenient for strict evaluation:  arguments can be evaluated in a tight
loop.

Our solution to this entails passing \emph{stgApply} the object to be applied
in \emph{CV}, together with an extra argument---call it
\emph{index}---indicating which of the object to be applied and its boxed
arguments have been already been evaluated (previously by \emph{stgApply}).
If the object to be applied, or some argument (because of the strictness
level), remains to be evaluated, it is jumped to, with return as ever through
the continuation stack.  This is actually fairly efficient because we can
arrange that the stack frame for \emph{stgApply} is reused, updating only
\emph{index}, so except for the indirect jumps on return this is essentially a
loop.  For the cases of arguments in excess of the arity of the object being
applied, in effect \emph{stgApply}'s argument frame is popped and two
continuations are pushed, the first one to jump to \emph{stgApply} with the
excess arguments, the next an argument frame for the object to applied, then
the code for the object to be applied is jumped to.
\end{comment}

\begin{comment}
\subsubsection{Is this too much for this paper?}
Similar to what is described in \emph{EA}, in an earlier
implementation we have a generic variadic \emph{stgApply} C function that is
used for applications of non-known functions, and known functions with a
deficit or excess number of the arguments.  The \emph{stgApply} function takes
as arguments the first element of the application (\emph{FUN}, \emph{PAP},
\emph{THUNK}, or \emph{BLACKHOLE}) and its arguments, and if a \emph{THUNK},
evaluates it with a call-with-return to yield a \emph{FUN} or \emph{PAP}.
This allows us to implement a single \emph{stgApply} function as a C function
taking a contiguous block of machine words (unboxed values or pointers) on the
explicit stack.  The implementation of \emph{stgApply} as described in
\emph{EA} performs the same call-with-return, but the scheme described there
uses an auto-generated family of \emph{stgApply} functions parameterized by
number and type (unboxed or pointer) of arguments.  (We used this scheme
initially as well, but because the stack is now fully explicit and the stack
frame is self-describing (e.g., contains a bitmap describing which arguments
are boxed/unboxed for the benefit of the garbage collector), this elaboration
is unnecessary.) Call-with-return was also used in the case that a \emph{FUN}
or \emph{PAP} is provided more than its arity number of arguments, in which
case \emph{FUN} or \emph{PAP} is called-with-return with its arity number
of objects, and the returned object and remaining argument passed (via jump)
to \emph{stgApply}.  Finally, for strict evaluation orders it is straightforward
to evaluate function or constructor arguments one at a time with call-with-return.

At that point of development these calls-with-return were clearly a hack
in an otherwise continuation-passing (via the explicit stack) control flow.
Because call-with-return could be of unbounded call depth this would be
problematic for stack unwinding.

The solution, slightly simplified, entailed adding a new parameter to \emph{stgApply}
and make the first parameter, the object to be applied, implicit.  The implicit
parameter is in \emph{CV}---a pointer to the object to be applied when jumping to \emph{stgApply}.
The new argument simply indicates which of the boxed arguments

For \emph{direct calls}, in which a known function is applied
to a number of arguments equal to its arity, \emph{stgApply} need not be
invoked: the \emph{FUN} is tail-called directly.


In addition to being in a tail call position, the gcc constraints for
generating a tail call (jump) are (approximately) that the total size of the
caller and callee arguments be equal, and the return type sizes be equal.  For
Clang the requirement is slightly more strict: caller and callee type
signatures must be the same.  These are a consequence of the C calling
convention that the caller, not callee, cleans up the stack (removes the
callee's stack frame).  However, because we maintain our own stack we can
define our own calling convention(s), so tail calls can be made to functions
of notionally differing type.  In particular, top-of-stack frames can be
adjusted in size and content without undue copying.
\end{comment}


\subsection{Mini-Haskell}
As mentioned, a current near-term goal is to use GHC as our front-end to
generate STG (or Core) as input to our system.  In the meantime, as a stop-gap
measure to make the generation of test cases easier, we have implemented a
\emph{mini-Haskell} front-end that supports a subset of standard Haskell
syntax with \emph{MagicHash} enabled, with a single extension.  Recall that
\emph{MagicHash} allows \texttt{\#} to be a suffix of an identifier, or denote
an unboxed literal value.  Our extension (to STG as well) is the provision of
the optional \texttt{unboxed} keyword in datatype definitions to allow
user-defined unboxed data types, e.g.,
\begin{verbatim}
  data unboxed Bool# = False# | True# .
\end{verbatim}

Writing STG in the concrete syntax conceived for our project is tedious and
prone to errors.  Compared to Haskell, the eventual source language for the
compiler, STG is relatively verbose, requiring explicit block and expression
delimiters, and classification of heap objects. Though its general structure
is similar to primitive Haskell, it lacks many of the conveniences that make
Haskell a pleasure to program in.  The motivation for a more expressive
language for testing our compiler and runtime system in their nascent stages
is then clear.  This language, mini-Haskell, was progressively ``grown'' from
the STG syntax towards a subset of Haskell itself, removing much of the
frustrating syntactic cruft and adding some of the core features that are
often taken for granted in Haskell (and other languages).

An informal specification of the mini-Haskell grammar is given in
Figure~\ref{fig:miniHaskell}.  There is significant overlap with STG resulting
from the structural similarities, but the redundancies are included for
completeness.


% Haskell Report:
%   https://www.haskell.org/onlinereport/haskell2010/haskell.html
% (PDF):
%   https://www.haskell.org/definition/haskell2010.pdf

% Chapter 10:
%   https://www.haskell.org/onlinereport/haskell2010/haskellch10.html
% (No standalone PDF)


% because # looks strangely large next to math terms
\newcommand{\hash}{{\scriptsize\#}}

\begin{figure}
\centering
\footnotesize % tiny scriptsize footnotesize small
\begin{tabular}{r r c l l}
Variable         & $f,\ x$             &     &
& Initial lower-case letter \\

Constructor        & $C$                 &     & 
& Initial upper-case letter \\

Literal            & $\mathit{lit}$      & ::= & $i\ |\ d\ 
                                                 |\ i$\hash\ $|\ d$\hash
& Integral or floating point literal (boxed) \\

                   &                     &     & 
& optional \# suffix denotes unboxed value \\

Atom               & $a$                 & ::= & $\mathit{lit}\ |\ x\ |\ C$
& Constructors are first class \\ 
% not sure if that's the right terminology 
\\


Expression         & $e$                 & ::= & $a$
& Atomic expression \\

                   &                     & $|$ & $e\ e$
& Expression application \\

                   &                     & $|$ & \texttt{\textbackslash} $\mathit{pat}_i$ 
                                               \texttt{->} $e$
& Lambda expression, $i > 0$ \\

                   &                     & $|$ & \texttt{case} $e$ \texttt{of} $\mathit{alt}_i$
& Case expression, $i > 0$\\

                   &                     & $|$ & \texttt{let} $\mathit{odecl}_i$ \texttt{in} $e$
& Recursive let expression, $i > 0$ \\

\\


Pattern            & $\mathit{pat}$      & ::= & $C\ \mathit{pat}_i$
& Nested constructor matching \\
                   &                     & $|$ & $\mathit{lit}$
& Match numeric literals \\
                   &                     & $|$ & $x$
& Bind to variable \\

\\


Alternative        & $\mathit{alt}$      & ::= & $\mathit{pat}_i$ \texttt{->} $e$
& $i > 0$ \\

\\


Type               & $\mathit{type}$     & $|$ & $C$
& Concrete type (constructor naming) \\
                   &                     & $|$ & $x$
& polymorphic type variable \\
                   &                     & $|$ & $\mathit{type}$ \texttt{->} $\mathit{type}$
& Function type \\

\\


Object decl. & $\mathit{odecl}$    & ::= & $x$ \texttt{::} $\mathit{type}$
& Type signature \\
                   &                     & $|$ & $x = e$
& Simple binding \\
                   &                     & $|$ & $f\ \mathit{pat}_i = e$
& Function declaration \\
                   &                     &     &
& $e$ cannot be an unboxed literal \\

\\


Constructor defn. & $\mathit{con}$  & ::= & $C\ \mathit{type}_i$
& $i \ge 0$ \\

Datatype defn. &  $\mathit{ddecl}$ & ::= & \texttt{data} 
                                                 [\texttt{unboxed}]
                                                 $C\ x_i =$ 
& User-defined data type, $i \ge 1, n > 0$ \\
                     &                   &     & $\mathit{con}_1 | \dots |
                                                 \mathit{con}_n$
& \\

\\


Program                & $\mathit{prog}$ & ::= & $(o|d)\mathit{decl}_i$
& Objects and datatypes

\end{tabular}
\label{fig:miniHaskell}
\caption{mini-Haskell syntax}
\end{figure}


Because mini-Haskell was designed to replace STG, it makes sense to talk about
it in terms of the differences between the two and how mini-Haskell is
translated into STG for the backend.  Here, the differences are given in order
of perceived utility, or importance to someone transition from STG to
mini-Haskell.

\subsubsection{Removal of explicit heap objects.}
In STG, all object definitions require explicit classification with
\texttt{FUN}, \texttt{PAP}, \texttt{CON}, \texttt{THUNK}. This classification can be
inferred from context by the compiler so these keywords are not necessary.
Functions can then be defined in Haskell style with the parameters following
the function's identifier on the left side of the equality. With these simple
changes the syntax becomes a subset of Haskell.

Heap objects are still only created at the top level or in \texttt{let}
expressions, so the translation is straightforward: \texttt{FUN} objects are
identifiable by the inclusion of these parameters.  \texttt{PAP} objects are
created when a known function is partially applied.  \texttt{CON} objects are
created when constructors are fully applied.  Every other object becomes a
\texttt{THUNK}.

\subsubsection{Avoiding building thunks.}
As \emph{EA} points out, there is no need to build a thunk for the scrutinee
of a \texttt{case} expression because if the \texttt{case} is evaluated the
scrutinee is certain to be evaluated.  Similarly, if a function is strict
there is no need to build a thunk for each of its arguments (assuming boxed
argument type).  Because STG allows only \emph{atoms} (variables or literal
values) in argument positions, naive translation of Haskell \texttt{f e} to
\texttt{let \{x = e\} in f x}, followed by naive code generation, will result in
an unnecessary heap allocation.  An alternative is to translate to
\texttt{case e of x \{\_ -> f x\}}.

\subsubsection{The Layout Rule}
Mini-Haskell implements Haskell's layout rule in an almost direct translation
of the specification given in section 10.3 of the Haskell 2010
Report.~\cite{haskell2010report} The notable exception is the omission of any
attempt to define the \texttt{parse-error} function.  This is apparently a
tricky spot for Haskell compilers (even GHC has not always had it right) so,
in the name of tradition, it remains a tricky spot for our mini-Haskell front
end.  The edge cases that it will incorrectly handle as a result of this are
exactly that: edge cases.  As such, they are easy to avoid in practical usage.

The inclusion of the layout rule may seem small and insignificant, but it
allows users to produce code that is visually similar to Haskell code.
This makes it easier to read and write for any Haskell programmer who relies
on layout to delineate context, presumably the vast majority.  It also allows
for the use of Haskell programming environments that support ``smart''
indentation, e.g., \emph{haskell-mode} for Emacs.~\cite{haskellmode}.



% Layout Rule Spec:
%   https://www.haskell.org/onlinereport/haskell2010/haskellch10.html\#x17-17800010.3
% GHC's syntax "infelicities":
%   https://downloads.haskell.org/~ghc/latest/docs/html/users\_guide/bugs-and-infelicities.html\#infelicities-syntax
% More on parse-error:
%   http://www.dcs.gla.ac.uk/mail-www/haskell/msg01711.html


\subsubsection{Expression to Expression application}

In STG, function application must be to atomic values, i.e., either literals or
variables.  This often requires introducing bindings with either \texttt{let} or
\texttt{case} expressions.  Consider the \texttt{append} function that
concatenates its two list parameters.  In STG, it is written as

\begin{verbatim}
append = FUN(l1 l2 ->
            case l1 of
              { Nil -> l2;
                Cons hd tl -> 
                  let { rec = THUNK(append tl l2);
                        result = CON(Cons hd rec) }
                  in result });
\end{verbatim}

In mini-Haskell, expressions may be applied to other expressions, so the
recursive call to \texttt{append} can be parenthesized. There is no need for the
\texttt{let} expression:

\begin{verbatim}
append l1 l2 = case l1 of
                Nil -> l2
                Cons hd tl -> Cons hd (append tl l2)
\end{verbatim}

This also serves as a good example of how expression to expression application is
transformed into STG.  Any non-atomic expression in an application expression is
bound to a heap object in a generated \texttt{let} expression.  These
expressions are thus properly ``atomized'' for STG.  The above mini-Haskell code
is transformed in STG that is structurally identical to the STG code that
precedes it.

\subsubsection{Fancy Pattern Matching.}
{\color{red}Implemented, not documented\dots}



\subsubsection{Less significant features}

\begin{itemize}
\item Constructors as functions \\
  (partial application uses generated functions, full creates CONS)

\item Primitive operations as functions \\
  (partial application uses generated functions, full creates EPrimop)

\item Lambda Expressions \\
  (desugar to let-bound FUN objects)

\item Type Annotations \\
  (passed on to type checker)

\item Haskell Block \\
  (ignored by MHS front-end, not other Haskell compilers;
  allows the same file to be tested on our compiler vs GHC)
\end{itemize}


\section{Current Status, and Expected Results by the time of TiFP'16}

Developing the bridge between GHC Core or STG to our system is scheduled
to start immediately after TiFP'16.

At the time of writing we have a complete serial implementation, including
\begin{itemize}
\item STG and mini-Haskell (inc. ADT w/unboxed) parsers \emph{(both by students)},
\item mini-Haskell to STG transformer \emph{(by student)},
\item type inferencer,
\item various analysis and transformation passes \emph{(most by students)},
\item code generator \emph{(some student contribution)},
\item runtime system including a garbage collector,
\item extensive test suite \emph{(mostly by students)}.
\end{itemize}

We have implemented a simple copying garbage collector~\cite{Cheney:1970} with
small \emph{nurseries} in the multithreaded case~\cite{Marlow:2008}. We have
not worked on the development of a more sophisticated garbage collector
because this would be a distraction from our main goals.

%Mutual exclusion for garbage collector--heap check with every heap allocation.
%In fact garbage collector can be invoked at any point in time except when
%populating a heap object immediately after its allocation.

\subsection{Type system}

We perform standard Hindley-Milner type inference and
enforcement~\cite{Heren02}, augmented with built-in and user-defined unboxed
types as described by Peyton-Jones and Launchbury~\cite{Jones:1991},
but for unboxed types code generation and runtime support is currently limited
to built-in single-word (64-bit) or smaller types (e.g., \texttt{Int\#},
\texttt{UInt\#}, \texttt{Float\#}, \texttt{Double\#}) and user-defined
enumerations, e.g., \texttt{data unboxed Bool\# = False\# | True\#}.

\subsection{Test suite and testing}

We have developed a extensive test suite which uses the CMake CTest
tool.~\cite{cmake-book} The full test suite is automatically run, and any
failures reported, after every commit to the Git revision control system. We
use a Jenkins continuous integration server to automate this build and testing
process.~\cite{jenkins-book} The Git repository is hosted externally so our
student collaborators can contribute at their leisure.

There are several hundred small STG and mini-Haskell programs in the test suite 
(Most of which were written by students).
It is partitioned into a number of directories which control what
testing is done for a specific program. Given our goal of exploring various
levels of strictness, most tests are run with all evaluation strategies (three
degrees of function strictness by two degrees of constructor strictness).
One interesting class of test programs are those which require lazy
semantics. For these programs we test not only that the test leads to the expected
result with non-strict semantics, but also that the program returns a
\emph{BLACKHOLE} in the case of strict semantics. Students who were Haskell/STG
beginners have contributed substantially to the test suite. We have also found
that the test suite has be of use as a learning tool for these students to
discover what level of non-strictness was required for their programs,
a benefit that we did not initially envision. Over time we have developed
extensive STG and mini-Haskell preludes of common programming patterns to
aid in the writing of test programs.

Another important class of test programs are those which should fail with a
known error (e.g. a parsing error) for these programs the test suite checks
the output against an expected error regular expression. This class of test
programs has proved valuable in development and debugging of both the STG
and mini-Haskell front-ends.

We also run each test with various levels of garbage collection enabled. This
has helped us to find bugs not only in the garbage collector, but also in the
code generator.  Interestingly, in practice garbage collection can both mask
and expose bugs.

The various enumerations of strictness and garbage collection levels mean that
we run a quite large number of tests (currently order thousands). However,
this is fully automated via the testing and continuous integration
infrastructure.

We have also implemented a simple logging facility that can be controlled at 
both compile and runtime which helps in debugging why a test is failing.

\subsection{Parallel evaluation}

Parallel runtime is being implemented following the ideas
of~\cite{Harris:2005,Marlow:2008,Marlow:2009}, thread pool, stack and heap 
for each thread, blah, blah.

\section{Related Work}

Various papers describing some of the inner workings of GHC at various points
in its evolution have greatly informed our approach and these have all been
previously cited.

\subsubsection{\emph{Strict but still pure} is not new.}
Very simple implementations have been prototyped, for example
Sheard's.~\cite{Sheard:2003}
%
A plug-in for GHC (prior to 8.0) exists for making functions, but not data
structures, strict using a transformation on GHC Core that recursively expands
non-recursive let bindings into case bindings.~\cite{strict-ghc-plugin}.  Thus
it seems that while it makes functions denotationally strict, function
arguments are still evaluated after function call.
%
The Disciplined Disciple Compiler project appears to intend implementation of
a strict, higher order functional language (or family of languages) compiler
that has been underway for some time, but the emphasis appears to be on a much
more complex type system for handling stateful computation, and the prospects
for its eventual completion are unknown~\cite{disciplined-disciple}.
%
NVIDIA's NOVA language, and corresponding compiler, implements a strict
(``call by value''), pure, polymorphic, higher-order functional language with
built-in parallel operators (e.g., \emph{map}, \emph{reduce}, and
\emph{scan})~\cite{NVIDIA:2013}. Three back-ends were implemented: for
sequential C, parallel (multi-threaded) C, and CUDA C, the lattermost
targeting NVIDIA GPUs.

\subsubsection{GHC as a front-end.}
The Intel Labs Haskell Research Compiler (HRC) is the most closely related
work to ours of which we are aware~\cite{Liu:2013,Petersen:2013}.  As in our
planned approach they use GHC as the front end, but exit the GHC toolchain
at the final GHC Core stage, just before transformation to STG.  The essence
of their effort is to target an existing, highly optimizing, Intel functional
language compiler that is ``largely language agnostic'' and performs (among
many other optimization) SIMD vectorization.

Don Stewart described an implementation wherein GHC was used as a Haskell compiler
front-end, dumping STG code as text, parsing that text, then translating to
code for execution on the Java Virtual Machine.~\cite{Stewart-BSc}.

\subsubsection{Fault tolerance for functional programming.}
Pointon et al.\ developed a limited fault tolerance mechanism for Glasgow
distributed Haskell based on distributed exception
handling~\cite{Pointon:2001}.  However, it appears that their scheme for
handling failed processing elements correctly~\cite{} may have never been
implemented.

Recently, Robert Stewart extended Haskell Distributed Parallel Haskell
(HdpH)~\cite{hdph} to be fault resilient at an impressive scale---1400 cores
on a distributed memory system.  Here the key recovery strategy is task
replication.~\cite{Stewart:2013}.

\section{Future Work}

An almost unbounded amount of interesting work suggests itself; one such path
would lead to the recreation of GHC.  However, given the minuscule budget (and
thus the emphasis on student labor) we must maintain sharp focus on the main
goals:  implementing and evaluating mechanisms for automatic parallelism and
fault tolerance.

Currently we are developing the parallel runtime.  This summer (2016) a
returning student will bridge the gap between GHC STG (or Core) and our STG,
and augment our runtime as needed in this effort.  Just as in the Intel
Haskell Research Compiler effort, we will not attempt to reuse the GHC
runtime.

\section{Acknowledgments}

Funding for this project has been provided by the DOE NNSA LANL Laboratory
Directed Research and Development program award 20150845ER; the National
Science Foundation Science, Technology, Engineering, and Mathematics Talent
Expansion Program (STEP) program for undergraduate students; and the DOE
Science Undergraduate Laboratory Internship (SULI) program.
%
Los Alamos National Laboratory is managed and operated by Los Alamos National
Security, LLC (LANS), under contract number DE-AC52-06NA25396 for the
Department of Energy's National Nuclear Security Administration (NNSA).


%
% ---- Bibliography ----
%

\bibliographystyle{splncs03}

\bibliography{tfp}

\end{document}
