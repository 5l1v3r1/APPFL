\documentclass[DIV16,twocolumn,10pt]{scrreprt}
\usepackage{paralist}
\usepackage{graphicx}
\usepackage[final]{hcar}

%include polycode.fmt

\begin{document}

\begin{hcarentry}{Auto-parallelizing Pure Functional Language System}
\report{Kei Davis}
\status{active}
\participants{Dean Prichard, David Ringo, Loren Anderson, Jacob Marks}
\makeheader

%(WHAT IS IT?)

The main project goal is the demonstration of a light-weight, higher-order,
polymorphic, pure functional language implementation in which we can
experiment with automatic parallelization strategies, varying degrees of
default function and constructor strictness.  A secondary goal is to
experiment with mechanisms for transparent fault tolerance.
%
We do not consider speculative or eager evaluation, or semantic strictness
inferred by program analysis, so potential parallelism is dictated by the
specified degree of default strictness and any strictness annotations.

Our approach is similar to that of the
\href{https://dl.acm.org/citation.cfm?id=2503779}{Intel Labs Haskell Research
  Compiler}: to use GHC as a front-end to generate STG (or Core), then exit to
our own back-end compiler that is itself written in Haskell.  As in their case
we do not attempt to use the GHC runtime.  Our implementation is
\emph{light-weight} in that we are not attempting to support or recreate the
vast functionality of GHC and its runtime.  This approach is also similar to
\href{http://www.cse.unsw.edu.au/~pls/thesis/dons-thesis.ps.gz}{Don Stewart's}
except that we generate C instead of Java.

%(WHAT IS ITS STATUS? / WHAT HAS HAPPENED SINCE LAST TIME?)
\subsubsection*{Current Status}
Currently we have a fully functioning serial implementation and a primitive
proof-of-design parallel implementation.

%(CAN OTHERS GET IT?)

%(WHAT ARE THE IMMEDIATE PLANS?)
\subsubsection*{Immediate Plans}

We are currently developing a more realistic parallel runtime.
%
Bridging the gap between GHC STG (or Core) to our STG representation will be
undertaken starting June 2016.  An instrumentation framework will be developed
in summer 2016.

\FurtherReading
 A \href{https://github.com/losalamos/APPFL}{project web site} is under construction.

\subsubsection*{Undergraduate/post-graduate Internships}

If you are a United States citizen or permanent resident alien studying
computer science or mathematics at the undergraduate level, or are a recent
graduate, with strong interests in Haskell programming, compiler/runtime
development, and pursuing a spring, fall, or summer internship at Los
Alamos National Laboratory, this could be for you.

We don't expect applicants to necessarily already be highly accomplished
Haskell programmers---such an internship is expected to be a combination of
further developing your programming/Haskell skills and putting them to good
use.  If you're already a strong C hacker we could use that too.

\emph{\bfseries The application process requires a bit of work so don't leave
  enquiries until the last day/month.}\\

\begin{tabular}{l|l}
Term & Application Deadline \\
\hline
Summer 2016  & Closed \\
Fall 2016    & May 31, 2016 \\
Spring 2017  & Approx. July 2016 \\
Summer 2017  & Approx. January 2017 \\
Fall 2017    & Approx. May 2017
\end{tabular}\\

\noindent Email me at kei (at) lanl (dot) gov if interested in more information, and
feel free to pass this along. \\

\end{hcarentry}
\end{document}
