
%% Kluwer Small Journal Article Template File
%% Current version: January 24, 1994

%%%%%%%%%%%%%%%%%%%%%%%%%%%%%%%%%%%%%%%%%%%%%%%%%%%%%%%%%%%%%%%%%%%%%%%%%%
%% Prepared by Amy Hendrickson, TeXnology Inc.                          %%
%% (617) 738-8029                                                       %%
%%                                                                      %%
%% Inquiries to Suzanne M. Rumsey, net address: prod@world.std.com      %%
%%%%%%%%%%%%%%%%%%%%%%%%%%%%%%%%%%%%%%%%%%%%%%%%%%%%%%%%%%%%%%%%%%%%%%%%%%


% Use one of these two commands: 
\documentstyle{smjrnl} % For Computer Modern Fonts, or,
% \documentstyle[smjfonts]{smjrnl} % for PostScript fonts

\begin{document}
\begin{article}


%%%%% To be entered at Kluwers: =====>>
\journame{}
\volnumber{}
\issuenumber{}
\issuemonth{}
\volyear{}

%% Do not delete either of the following two commands.
%% Please supply facing curly brackets for the part you
%% are not using for this article.
\received{May 1, 1991}\revised{}

\authorrunninghead{}
\titlerunninghead{}

%\setcounter{page}{275} %% Optional, uncomment to set page number.
%%%%  <<==== End of commands to be entered at Kluwers 


%%  Authors, start here ====>>

\title{}

%% Author name 
\authors{}

%% Email address. Do not delete this command even if you do not have
%% an email address; just leave the command followed by facing curly brackets.
\email{} 

%% Affiliation address
\affil{}

%% Article editor
\editor{}

%% Abstract
\abstract{}

%% Keywords
\keywords{}

\section{}


%% End matter:

% \acknowledgements

% \appendix{}

% \begin{references}
% \bibitem{xxx}
% \end{references}

%% Do not delete this! ===>>>
\end{article}
\end{document}


% Samples of commands you may use:

\section{ }
\subsection{ }
\subsubsection{ }

%% If you want to make an equation that is not indented, use
%% \begin{wideequation} ... \end{wideequation}

% \begin{wideequation}
% \begin{equation} or \[
% <math>
% \end{equation} or \]
% \end{wideequation}

% In this wide equation the `array' command is used to split 
% the math into two lines, moving the top half to the left
% and the bottom to the right.
% \begin{wideequation}
% \begin{array}{lr}
% \sum_k P(k) \sum_i \sum_y f_i(y|k)^2\\
% &\sum_k P(k) \sum_i \sum_y f_i(y|k)^2
% \sum_k P(k) \sum_i \sum_y f_i(y|k)^2
% \end{array}
% \end{wideequation}


% To indent text, use the following commands:
% \begin{itemize}
% \item[] 
% text...
% \end{itemize}


%% Algorithm for exhibiting code. Indent lines with one or more `\ '.
% \begin{algorithm}
% \ start line here
% \ \ indent line here
% \end{algorithm}


% Sample figure
% \begin{figure}[h]
% \vspace*{.5in}
% \caption{This is a figure caption.
% This is a figure caption.
% This is a figure caption.}
% \end{figure}

% Sample table, this kind of table preamble will spread 
% table out to the width of the page:
% \begin{table}[h]
% \caption{This is an example table caption. As you can
% see, it will be as wide as the table that it captions.}
% \begin{tabular*}{\textwidth}{@{\extracolsep{\fill}}lcr}
% \hline
% $\alpha\beta\Gamma\Delta$ One&Two&Three\cr
% \hline
% one&two&three\cr
% one&two&three\cr
% \hline
% \end{tabular*}
% \end{table}

%% Make examples:
% \begin{example}
% text...
% \end{example}

%% Make theorems:
% \begin{proclaim}{Theorem <number>}
% text...
% \end{proclaim}

%% Make proof:
% \begin{proof}
% text...
% \end{proof}

% If proof ends with math, please use \inmathqed at the end of the
% equation:
% \begin{proof}
% ....
% \[
% \alpha\beta\Gamma\Delta\inmathqed
% \]
% \end{proof}

%% Proof with a title. Enter name of proof in square brackets:
% \begin{proof}[Proof of Theorem A.1]
% ...
% \end{proof}

%% Assumption or similar kind of environment
% \begin{demo}{Assumption <number>}
% text...
% \end{demo}

%% End Matter:

%%%%%%%
%% Acknowledgements here

% \acknowledgements
% text...

% Appendices:
% \appendix

%%%%%%%
%% Endnotes

%\notes

%%%%%%%
%% Make references as standard Latex.
%% for example:

%% Trying `cite', \cite{jacobs}, \cite{francis}.

% \begin{references}
% \bibitem{jacobs}Jacobs, E., ``Design Method Optimizes Scanning
% Phased Array,'' Microwaves, April 1982, pp.\ 69--70.

% \bibitem{francis} Francis, M., ``Out-of-band response of array antennas,''
% Antenna Meas.  Tech. Proc., September 28--October 2, 1987, Seattle, p.~14.
% \end{references}

% Or, to make alphabetical references, with no number preceding entries:

% Maude Francis, (Francis, 87) showed important new results
% with array antennas.

% \begin{alphareferences}
% Francis, M., ``Out-of-band response of array 
% antennas,'' Antenna Meas.  Tech. Proc., September 28--October 2,
% 1987, Seattle, p.~14.

% Jacobs, E., ``Design Method Optimizes Scanning
% Phased Array,'' Microwaves, April 1982, pp.\ 69--70.
% \end{alphareferences}

See smjrnl.doc for documentation on using Bibtex,
also on making an alphabetical reference section using Bibtex.




