% AutoparallelizingPureFunc-KA.tex
\begin{hcarentry}[new]{Auto-parallelizing Pure Functional Language System}
\report{Kei Davis}%05/16
\status{active}
\participants{Dean Prichard, David Ringo, Loren Anderson}
\makeheader

The main project goal is the demonstration of a light-weight, higher-order,
polymorphic, pure functional language implementation in which we can
experiment with automatic parallelization strategies and varying degrees of
default function and constructor strictness.  A secondary goal is to
experiment with mechanisms for transparent fault tolerance.

We do not consider speculative or eager evaluation, or semantic strictness
inferred by program analysis, so potential parallelism is dictated by the
specified degree of default strictness and any strictness annotations.

Our approach is similar to that of the
\href{https://dl.acm.org/citation.cfm?id=2503779}{Intel Labs Haskell
Research Compiler}, using GHC as a front-end to generate STG, then
exit to our own back-end compiler.  As in their case we do not attempt to use
the GHC runtime.  Our implementation is \emph{light-weight} in that we are not
attempting to support or recreate the vast functionality of GHC and its
runtime.  This approach is also similar to
\href{http://www.cse.unsw.edu.au/~pls/thesis/dons-thesis.ps.gz}{Don Stewart's}
except that we generate C instead of Java.

\subsubsection*{Current Status}
Currently we have a fully functioning serial implementation and a primitive
proof-of-design parallel implementation.  The most recent major development
was the ``bridge'' between GHC and our system.  Thus we can now compile and run
Haskell programs with simple primitive and algebraic data types.

\subsubsection*{Immediate Plans}

We are currently developing a more realistic parallel runtime.  Tentatively
the fault tolerance mechanism is scheduled as a Master's thesis project starting
summer 2017.

\subsubsection*{Undergraduate/post-graduate Internships}

If you are a United States citizen or permanent resident alien studying
computer science or mathematics at the undergraduate level, or are a recent
graduate, with strong interests in Haskell programming, compiler/runtime
development, and pursuing a spring, fall, or summer internship at Los Alamos
National Laboratory, this could be for you.

We don't expect applicants to necessarily already be highly accomplished
Haskell programmers---such an internship is expected to be a combination of
further developing your programming/Haskell skills and putting them to good
use.  If you're already a strong C hacker we could use that too.

\emph{\bfseries The application process requires a bit of work so don't leave
enquiries until the last day/month.}  Dates for terms beyond summer 2017 are
best guesses based on prior years.\\

\begin{tabular}{l||l|l}
Term & Application Opening & Deadline \\
\hline
Summer 2017  & Open & Jan 2017 \\
Fall 2017    & Jan 2017 & May 2017 \\
Spring 2018  & May 2017 & Jul 2017 \\
\end{tabular}\\

\noindent Email me at kei (at) lanl (dot) gov if interested in more
information, and feel free to pass this along. \\

\FurtherReading
Email me as above for the Trends in Functional Programming 2016 paper
about this project.

\end{hcarentry}
